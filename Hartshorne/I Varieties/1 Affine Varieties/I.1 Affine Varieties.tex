\documentclass[12pt,letterpaper]{article}
\usepackage{geometry}
\geometry{letterpaper}
\usepackage{amsmath,amssymb,amsthm,mathrsfs}
\usepackage{mathtools}
\usepackage{ifpdf}
  \ifpdf
    \setlength{\pdfpagewidth}{8.5in}
    \setlength{\pdfpageheight}{11in}
  \else
\fi
\usepackage{hyperref}

\usepackage{tikz}
\usepackage{tikz-cd}
\usetikzlibrary{decorations.markings}
\tikzset{
  open/.style = {decoration = {markings, mark = at position 0.5 with { \node[transform shape] {\tikz\draw[fill=white] (0,0) circle (.3ex);}; } }, postaction = {decorate} },
  closed/.style = {decoration = {markings, mark = at position 0.5 with { \node[transform shape, xscale = .8, yscale=.4] {\upshape{/}}; } }, postaction = {decorate} },
  imm/.style = {decoration = {markings, mark = at position 0.3 with { \node[transform shape, xscale = .8, yscale=.4] {\upshape{/}}; }, mark = at position 0.6 with { \node[transform shape] {\tikz\draw[fill=white] (0,0) circle (.3ex);}; } }, postaction = {decorate} }
}

\usepackage{braket}

\usepackage{csquotes}
\usepackage[american]{babel}
\usepackage[style=alphabetic,firstinits=true,backend=biber,texencoding=utf8,bibencoding=utf8]{biblatex}
\bibliography{../../../References}
\AtEveryBibitem{\clearfield{url}}
\AtEveryBibitem{\clearfield{doi}}
\AtEveryBibitem{\clearfield{issn}}
\AtEveryBibitem{\clearfield{isbn}}
\renewbibmacro{in:}{}
\DeclareFieldFormat{postnote}{#1}
\DeclareFieldFormat{multipostnote}{#1}

\renewcommand{\theenumi}{$(\alph{enumi})$}
\renewcommand{\labelenumi}{\theenumi}

\newcounter{enumacounter}
\newenvironment{enuma}
{\begin{list}{$(\alph{enumacounter})$}{\usecounter{enumacounter} \parsep=0em \itemsep=0em \leftmargin=2.75em \labelwidth=1.5em \topsep=0em}}
{\end{list}}
\newcounter{enumdcounter}
\newenvironment{enumd}
{\begin{list}{$(\arabic{enumdcounter})$}{\usecounter{enumdcounter} \parsep=0em \itemsep=0em \leftmargin=1.75em \labelwidth=1.5em \topsep=0em}}
{\end{list}}
\newtheorem*{theorem}{Theorem}
\newtheorem*{universalproperty}{Universal Property}
\newtheorem{problem}{Exercise}[section]
\newtheorem{subproblem}{Problem}[problem]
\newtheorem{lemma}{Lemma}%[section]
\newtheorem{proposition}{Proposition}
\newtheorem{property}{Property}[problem]
\newtheorem*{lemma*}{Lemma}
\theoremstyle{definition}
\newtheorem*{definition}{Definition}
\newtheorem*{claim}{Claim}
\theoremstyle{remark}
\newtheorem*{remark}{Remark}

\numberwithin{equation}{section}
\numberwithin{figure}{problem}
\renewcommand{\theequation}{\arabic{section}.\arabic{equation}}

\DeclareMathOperator{\Ann}{Ann}
\DeclareMathOperator{\Ass}{Ass}
\DeclareMathOperator{\Supp}{Supp}
\DeclareMathOperator{\WeakAss}{\widetilde{Ass}}
\let\Im\relax
\DeclareMathOperator{\Im}{im}
\DeclareMathOperator{\Spec}{Spec}
\DeclareMathOperator{\SPEC}{\mathbf{Spec}}
\DeclareMathOperator{\Sp}{sp}
\DeclareMathOperator{\maxSpec}{maxSpec}
\DeclareMathOperator{\Hom}{Hom}
\DeclareMathOperator{\Soc}{Soc}
\DeclareMathOperator{\Ht}{ht}
\let\AA\relax
\DeclareMathOperator{\AA}{\mathbb{A}}
\DeclareMathOperator{\V}{\mathbf{V}}
\DeclareMathOperator{\Aut}{Aut}
\DeclareMathOperator{\Char}{char}
\DeclareMathOperator{\Frac}{Frac}
\DeclareMathOperator{\Proj}{Proj}
\DeclareMathOperator{\stimes}{\text{\footnotesize\textcircled{s}}}
\DeclareMathOperator{\End}{End}
\DeclareMathOperator{\Ker}{Ker}
\DeclareMathOperator{\Coker}{coker}
\DeclareMathOperator{\LCM}{LCM}
\DeclareMathOperator{\Div}{Div}
\DeclareMathOperator{\id}{id}
\DeclareMathOperator{\Cl}{Cl}
\DeclareMathOperator{\dv}{div}
\DeclareMathOperator{\Gr}{Gr}
\DeclareMathOperator{\pr}{pr}
\DeclareMathOperator{\trd}{tr.d.}
\DeclareMathOperator{\rank}{rank}
\DeclareMathOperator{\codim}{codim}
\DeclareMathOperator{\sgn}{sgn}
\DeclareMathOperator{\GL}{GL}
\newcommand{\GR}{\mathbb{G}\mathrm{r}}
\newcommand{\gR}{\mathrm{Gr}}
\newcommand{\EE}{\mathscr{E}}
\newcommand{\FF}{\mathscr{F}}
\newcommand{\GG}{\mathscr{G}}
\newcommand{\HH}{\mathscr{H}}
\newcommand{\II}{\mathscr{I}}
\newcommand{\LL}{\mathscr{L}}
\newcommand{\MM}{\mathscr{M}}
\newcommand{\OO}{\mathcal{O}}
\newcommand{\Ss}{\mathscr{S}}
\newcommand{\Af}{\mathfrak{A}}
\newcommand{\Aa}{\mathscr{A}}
\newcommand{\PP}{\mathcal{P}}
\newcommand{\red}{\mathrm{red}}
\newcommand{\Sh}{\mathfrak{Sh}}
\newcommand{\Psh}{\mathfrak{Psh}}
\newcommand{\LRS}{\mathsf{LRS}}
\newcommand{\Sch}{\mathfrak{Sch}}
\newcommand{\Var}{\mathfrak{Var}}
\newcommand{\Rings}{\mathfrak{Rings}}
\DeclareMathOperator{\In}{in}
\DeclareMathOperator{\Ext}{Ext}
\DeclareMathOperator{\Spe}{Sp\acute{e}}
\DeclareMathOperator{\HHom}{\mathscr{H}\!\mathit{om}}
\newcommand{\isoto}{\overset{\sim}{\to}}
\newcommand{\isolongto}{\overset{\sim}{\longrightarrow}}
\newcommand{\Mod}{\mathsf{mod}\mathchar`-}
\newcommand{\MOD}{\mathsf{Mod}\mathchar`-}
\newcommand{\gr}{\mathsf{gr}\mathchar`-}
\newcommand{\qgr}{\mathsf{qgr}\mathchar`-}
\newcommand{\uqgr}{\underline{\mathsf{qgr}}\mathchar`-}
\newcommand{\qcoh}{\mathsf{qcoh}\mathchar`-}
\newcommand{\Alg}{\mathsf{Alg}\mathchar`-}
\newcommand{\coh}{\mathsf{coh}\mathchar`-}
\newcommand{\vect}{\mathsf{vect}\mathchar`-}
\newcommand{\imm}[1][imm]{\hspace{0.75ex}\raisebox{0.58ex}{%
\begin{tikzpicture}[commutative diagrams/every diagram]
\draw[commutative diagrams/.cd, every arrow, every label,hook,{#1}] (0,0ex) -- (2.25ex,0ex);
\end{tikzpicture}}\hspace{0.75ex}}
\newcommand{\dashto}[2]{\smash{\hspace{-0.7em}\begin{tikzcd}[column sep=small,ampersand replacement=\&] {#1} \rar[dashed] \& {#2} \end{tikzcd}\hspace{-0.7em}}}

\usepackage{todonotes}

\title{Hartshorne Ch.~I, \S1 Affine Varieties}
\author{Takumi Murayama}

\begin{document}
\maketitle
\setcounter{section}{1}
\begin{problem}\mbox{}\label{exc:I.1.1}
  \begin{enuma}
    \item Let $Y$ be the plane curve $y = x^2$ (i.e., $Y$ is the zero set of
      the polynomial $f = y - x^2$).
      Show that $A(Y)$ is isomorphic to a polynomial ring in one variable over
      $k$.
    \item Let $Z$ be the plane curve $xy = 1$.
      Show that $A(Z)$ is not isomorphic to a polynomial ring in one variable
      over $k$.
    \item Let $f$ be any irreducible quadratic polynomial in $k[x,y]$, and let
      $W$ be the conic defined by $f$.
      Show that $A(W)$ is isomorphic to $A(Y)$ or $A(Z)$.
      Which one is it when?
  \end{enuma}
\end{problem}
\begin{proof}[Proof of $(a)$]

  Mapping $k[x,y] \to k[t]$ by $x \mapsto t,y \mapsto t^2$, we have a
  surjective map with kernel $(y - x^2)$, and so $k[x,y]/(y-x^2) \cong k[t]$.
  Thus, $(y-x^2)$ is a prime ideal since $k[t]$ is a domain, and in particular
  it is radical; thus $I(Y) = (y-x^2)$, and $A(Y) = k[x,y]/(y-x^2) \cong k[t]$.
\end{proof}
\begin{proof}[Proof of $(b)$]
  $k[x,y]/(xy-1) \cong k[s,s^{-1}]$ by the map $x \mapsto s,y \mapsto s^{-1}$,
  which is a domain, hence $(xy-1)$ is prime, and in particular it is radical.
  Thus, $I(Z) = (xy-1)$, and $A(Z) \cong k[s,s^{-1}]$.
  This is not isomorphic to a polynomial ring, for any homomorphism
  $\varphi\colon k[s,s^{-1}] \to k[t]$ must have $\varphi(r) \in k$ for any
  $r \in k$ or $r \in \{s,s^{-1}\}$ since they are units, and so
  $\varphi(k[s,s^{-1}]) \subset k$.
\end{proof}
\begin{claim}[$c$]
  Let $\Char k \ne 2$, and let
  $f = a_{11}x^2 + a_{12}xy + a_{22}y^2 + b_1x + b_2y + c$,
  where is irreducible. Consider the matrix
  \begin{equation*}
    M = \begin{pmatrix}
      a_{11} & \dfrac{a_{12}}{2}\\
      \dfrac{a_{12}}{2} & a_{22}
    \end{pmatrix}
  \end{equation*}
  Then, $A(W) \cong A(Y)$ if $\det M = 0$, and $A(W) \cong A(Z)$ if
  $\det M \ne 0$.
  \par If $\Char k =2$, then $A(W) \cong A(Y)$ if $a_{12} = 0$, and
  $A(W) \cong A(Z)$ if $a_{12} \ne 0$.
\end{claim}
\begin{proof}[Proof of $(c)$]
  By the structure theorem for symmetric bilinear forms
  \cite[Ch.~XV, Thm.~3.1]{Lan02}, we can assume after change of coordinates
  in $\AA^2$ that $a_{12} = 0$.
  Since (after possibly interchanging $x \leftrightarrow y$) $a_{11} \ne 0$,
  changing coordinates $x \mapsto x - b_1/2a_{11}$ cancels out the $x$ term in
  $f$.
  If $a_{22} \ne 0$, we can do the same process to $y$ to eliminate the $y$
  term; otherwise, we can eliminate the $c$ term by replacing
  $y \mapsto y - c/b_2$.
  Thus, up to change in coordinates $f$ has one of the following two forms:
  \begin{equation*}
    a_{11}x^2 + a_{22}y^2 + c, \qquad a_{11}x^2 + b_2y.
  \end{equation*}
  We note in the first case that $c \ne 0$ for otherwise $f$ would be
  reducible (recalling that $k$ is algebraically closed), and that this
  corresponds to the case when $\det M \ne 0$.
  We note in the second case that $a_{11} \ne 0$ for $f$ was assumed to be
  quadratic, and $b_2 \ne 0$ for otherwise $f$ would be reducible; this
  corresponds to the case when $\det M = 0$.
  \par It remains to show the isomorphisms in either case.
  Note that since none of our coefficients above are zero, we can re-normalize
  our variables to get $x^2 - y^2 - 1$ and $x^2 - y$ (recalling again that $k$
  is algebraically closed).
  In the latter case, we clearly have that $A(W) \cong A(Y)$, and so the former
  case remains.
  In the former case, substituting $x \mapsto (x+y)/2$ and $y \mapsto (x-y)/2$
  gives that $f$ is of the form $xy - 1$, and so we have that $A(W) \cong A(Z)$.
  \par Finally, we show the claim for when $\Char k = 2$. First consider the
  case $a_{12} \ne 0$. Let $\alpha$ such that
  $\alpha^2 + a_{12}\alpha + a_{22} = 0$, which exists since $k$ is
  algebraically closed.
  Then, changing coordinates $x \mapsto x - \alpha y$,
  \begin{equation*}
    a_{11}x^2 + a_{12}xy + a_{22}y^2 \mapsto a_{11}x^2 + \alpha y^2 + a_{12}xy + a_{12}\alpha y^2 + a_{22}y^2 = a_{11}x^2 + a_{12}xy,
  \end{equation*}
  and so we can assume $a_{22} = 0$.
  Then, replacing $y \mapsto y + (a_{11}/a_{12})x$, we get
  \begin{equation*}
    a_{11}x^2 + a_{12}xy \mapsto a_{11}x^2 + a_{12}xy + a_{11}x^2 = a_{12}xy,
  \end{equation*}
  and so we can moreover assume $a_{11} = 0$.
  Finally, replacing $x \mapsto x - (b_2/a_{12}$ and
  $y \mapsto y - (b_1/a_{12})$, which we note are $k$-algebra isomorphisms,
  we get $f \mapsto a_{12}xy + c$ for some $c$.
  If $c = 0$ then $f$ was originally the union of two lines up to affine
  transformation, hence we have $c \ne 0$.
  So, $A(W)$ is isomorphic to $k[x,y]/(xy - 1) \cong A(Z)$ after rescaling
  of $x,y$.
  \par Now consider when $a_{12} = 0$; then, one of $a_{11},a_{22} \ne 0$.
  Suppose both are nonzero; then, changing coordinates
  $x \mapsto x + \sqrt{a_{22}/a_{11}}y$, we have
  \begin{equation*}
    a_{11}x^2 + a_{22}y^2 \mapsto a_{11}x^2 + a_{22}y^2 + a_{22}y^2 = a_{11}x^2,
  \end{equation*}
  and so we can assume $a_{22} = 0$, i.e., $f = a_{11}x^2 + b_1x + b_2y + c$.
  If $b_2 = 0$, then $f$ is a reducible polynomial defining one or two lines,
  a contradiction, so $b_2 \ne 0$.
  Then, changing coordinates $y \mapsto y + (b_1/b_2)x$, we have
  \begin{equation*}
    b_1x + b_2y \mapsto b_1x + b_2y + b_1x = b_2y,
  \end{equation*}
  and so we can assume $b_1 = 0$.
  Changing coordinates again such that $x \mapsto x + \sqrt{c}$,
  \begin{equation*}
    a_{11}x^2 + b_2y + c \mapsto a_{11}x^2 + c + b_2y + c = a_{11}x^2 + b_2y,
  \end{equation*}
  so renormalizing gives $f$ in the form $y - x^2$.
  Finally, this implies that $A(Z) = k[x,y]/(y-x^2) \cong A(Y)$.
\end{proof}

\begin{problem}[The Twisted Cubic Curve]
  Let $Y \subseteq \AA^3$ be the set $Y = \{(t,t^2,t^3) \mid t \in k\}$.
  Show that $Y$ is an affine variety of dimension $1$.
  Find generators for the ideal $I(Y)$.
  Show that $A(Y)$ is isomorphic to a polynomial ring in one variable over $k$.
  We say that $Y$ is given by the \emph{parametric representation} $x =t$,
  $y = t^2$, $z = t^3$.
\end{problem}
\begin{proof}
  We first show $\mathfrak{a} = (y - x^2,z - x^3)$ is prime and in particular,
  radical in $k[x,y,z]$. Define the $k$-algebra homomorphism
  $k[x,y,z] \to k[t]$ by $x \mapsto t,y\mapsto t^2,z \mapsto t^3$; clearly the
  kernel of this map contains $\mathfrak{a}$, and so this induces a map
  $\varphi\colon k[x,y,z]/\mathfrak{a} \to k[t]$.
  We claim $\varphi$ is injective. For suppose $f +\mathfrak{a} \mapsto 0$.
  By replacing $z$ with $xy$ and $y$ with $x^2$ in the expression for a
  particular representative $f$, $f + \mathfrak{a}$ is equivalent to a
  polynomial purely in $x\bmod\mathfrak{a}$.
  But any polynomial in $x$ maps to $0$ through $\varphi$ if and only if it is
  zero, hence $\varphi$ is injective.
  Since it was defined as the quotient of a surjective map, $\varphi$ is also
  an isomorphism of $k$-algebras, and so $\mathfrak{a}$ is prime and also
  radical since $k[t]$ is a domain.
  \par Now we claim $I(Y) = \mathfrak{a}$.
  $\supset$ is trivial since if $f \in \mathfrak{a}$, then $f(t,t^2,t^3) = 0$
  for any $t \in k$ since the relations $y - x^2,z - x^3$ are all satisfied
  when $x=t,y=t^2,z=t^3$.
  $\subset$. We first show $Y \supset Z(\mathfrak{a})$.
  Suppose $(x,y,z)$ satisfy $y - x^2,z - x^3$.
  Then, $(x,y,z) = (x,x^2,x^3)$ by using the relations repeatedly, and so
  $(x,y,z) \in Y$.
  Now by Prop.~$1.2$, $Y \supset Z(\mathfrak{a})$ implies
  $I(Y) \subset I(Z(\mathfrak{a})) = \sqrt{\mathfrak{a}} = \mathfrak{a}$
  since $\mathfrak{a}$ is radical from above. 
  \par We check that $Y$ is closed; however,
  $\overline{Y} = Z(I(Y)) = Z(\mathfrak{a})$, and any point $(x,y,z)$
  satisfying $y - x^2,z - x^3$ can be written $(x,x^2,x^3)$ as above, hence
  $Y \supset Z(\mathfrak{a})$, and so $Y = \overline{Y}$ is closed.
  \par Finally, by the first paragraph,
  $A(Y) = k[x,y,z]/I(Y) = k[x,y,z]/\mathfrak{a} \cong k[t]$, and so since
  $I(Y) = \mathfrak{a}$ is prime by the above with $Z(\mathfrak{a}) = Y$ and
  $k[t]$ is of dimension $1$, $Y$ is an affine variety of dimension $1$ by
  Cor.~1.4 and Prop.~1.7.
\end{proof}

\begin{problem}
  Let $Y$ be the algebraic set in $\AA^3$ defined by the two polynomials
  $x^2 - yz$ and $xz - x$.
  Show that $Y$ is a union of three irreducible components.
  Describe them and find their prime ideals.
\end{problem}
\begin{proof}
  We first know $Y = Z(x^2-yz,x(z-1))$, and so we claim
  \begin{equation*}
    Y = Z(x^2-yz,x(z-1)) = Z(yz,x) \cup Z(x^2-y,z-1).
  \end{equation*}
  $\subset$. If $P = (x,y,z) \in Z(x^2-yz,x(z-1))$, then either $x=0$ or
  $z-1=0$.
  In the former case, this implies $yz =0$ as well, hence $P \in Z(yz,x)$.
  In the latter case, this implies $x^2-y$ as well, hence $P \in Z(x^2-y,z-1)$.
  $\supset$. If $P \in Z(yz,x)$, then clearly $x^2-yz$ and $x(z-1)$ are
  satisfied; similarly, if $P \in Z(x^2-y,z-1)$, then $x^2-yz$ and $z-1$ are
  clearly satisfied.
  \par We now claim that $Z(yz,x) = Z(y,x) \cup Z(z,x)$. $\subset$ is clear
  since $x=0$ and one of $y = 0$ or $z=0$ must hold, so any point
  $P \in Z(yz,x)$ is in either $Z(y,x)$ or $Z(z,x)$.
  $\supset$ is trivial.
  \par We therefore have
  \begin{equation*}
    Y = Z(x,y) \cup Z(x,z) \cup Z(x^2-y,z-1).
  \end{equation*}
  We claim each component is irreducible.
  It suffices to show by Cor.~1.4 that each of the defining ideals is prime.
  $(x,y)$ is prime since $k[x,y,z]/(x,y) \cong k[z]$; similarly, $(x,z)$ is
  prime since $k[x,y,z]/(x,z) \cong k[y]$.
  $x^2-y,z-1$ is also prime since
  $k[x,y,z]/(x^2-y,z-1) \cong k[x,y]/(x^2-y) \cong k[t]$ as in Problem
  $\ref{exc:I.1.1}(a)$.
  \par We finally note that $Z(x,y)$ is the $z$-axis in $\AA^3$, $Z(x,z)$ is
  the $y$-axis in $\AA^3$, and $Z(x^2-y,z-1)$ is the parabola $x^2-y$ lying on
  the $z=1$ hyperplane.
\end{proof}

\begin{problem}
  If we identify $\AA^2$ with $\AA^1 \times \AA^1$ in the natural way, show
  that the Zariski topology on $\AA^2$ is not the product topology of the
  Zariski topologies on the two copies of $\AA^1$.
\end{problem}
\begin{proof}
  $Z(x-y) \subset \AA^2$ is closed. Now suppose $\AA^2$ had the product
  topology; then, by \cite[Exc.~17.13]{Mun00}, this would imply $\AA^1$ is
  Hausdorff, contradicting Ex.~1.1.1.
\end{proof}

\begin{problem}
  Show that a $k$-algebra $B$ is isomorphic to the affine coordinate ring of
  some algebraic set in $\AA^n$, for some $n$, if and only if $B$ is a
  finitely generated $k$-algebra with no nilpotent elements.
\end{problem}
\begin{proof}
  $\Leftarrow$. Suppose $B$ is a finitely generated $k$-algebra with no
  nilpotent elements.
  By definition of finitely-generated $k$-algebras,
  $B \cong k[x_1,\ldots,x_n]/I$ for some ideal
  $I \subset k[x_1,\ldots,x_n] \eqqcolon A$.
  Since $B$ has no nilpotent elements,
  $\sqrt{I} = \pi^{-1}(\mathfrak{N}_{A/I}) = I$ so
  $A(Z(I)) = A/I(Z(I)) = A/\sqrt{I} = A/I = B$ by Prop.~1.2.
  \par $\Rightarrow$. Suppose $B \cong A(X)$ for some algebraic set
  $X \subset \AA^n$.
  Then, $B \cong A/I(X)$, where $A$ is as before, hence $B$ is a finitely
  generated $k$-algebra.
  Now suppose $f + I(X) \in B$ were nilpotent, i.e., $(f + I(X))^r = 0$ for
  some $r$.
  Then, $f^r + I(X) = 0$, hence $f^r \in I(X)$ and so $f \in I(X)$ by
  Cor.~1.4 since $I(X)$ is radical, and so $f = 0$ in $B$.
  Thus, $B$ has no nilpotent elements.
\end{proof}

\begin{problem}\label{exc:1.6}
  Any nonempty open subset of an irreducible topological space is dense and
  irreducible.
  If $Y$ is a subset of a topological space $X$, which is irreducible in its
  induced topology, then the closure $\overline{Y}$ is also irreducible.
\end{problem}
\begin{remark}
  By definition, $X$ is irreducible if and only if in any decomposition
  $X = A \cup B$ for $A,B$ closed subsets of $X$, we have either $X = A$ or
  $X = B$.
\end{remark}
\begin{proof}
  Let $U \subset X$ be a nonempty open subset.
  Then, $X = (X \setminus U) \cup \overline{U}$, and since $X \setminus U$ is
  a proper subset of $X$, irreducibility of $X$ implies $X = \overline{U}$.
  \par Now let $U \subset X$ be an open subset of $X$, and let $A,B \subset X$
  be closed such that $U = (A \cap U) \cup (B \cap U) = (A \cup B) \cap U$.
  By the previous paragraph,
  $X = \overline{U} = \overline{(A \cup B) \cap U} = A \cup B$, hence by
  irreducibility of $X$, without loss of generality $A \supset U$, and so
  $U = U \cap A$, i.e., $U$ is irreducible.
  \par Finally, suppose $\overline{Y} = A \cup B$ for $A,B$ closed in
  $\overline{Y}$.
  Then, $Y = (Y \cap A) \cup (Y \cap B)$, and since $Y$ is irreducible, without
  loss of generality $Y = Y \cap A$.
  Then, $\overline{Y} = \overline{Y \cap A} = A$, hence $\overline{Y}$ is
  irreducible.
\end{proof}

\begin{problem}\mbox{}
  \begin{enuma}
    \item Show that the following conditions are equivalent for a topological
      space $X$: $(i)$ $X$ is noetherian; $(ii)$ every nonempty family of
      closed subsets has a minimal element; $(iii)$ $X$ satisfies the ascending
      chain condition for open subsets; $(iv)$ every nonempty family of open
      subsets has a maximal element.
    \item A noetherian topological space is \emph{quasi-compact,} i.e., every
      open cover has a finite subcover.
    \item Any subset of a noetherian topological space is noetherian in its
      induced topology.
    \item A noetherian space which is also Hausdorff must be a finite set with
      the discrete topology.
  \end{enuma}
\end{problem}
\begin{proof}[Proof of $(a)$]
  $(i) \Rightarrow (ii)$. If we had a nonempty family of closed subsets with
  no minimal element, then we could inductively construct a non-terminating
  strictly decreasing sequence of closed subsets of $X$, contradicting that
  $X$ is noetherian.
  \par $(ii) \Rightarrow (i)$. Any descending chain of closed subsets
  $Y_1 \supset Y_2 \supset \cdots$ has a minimal element $Y_r$, i.e.,
  $Y_r = Y_{r+1} = \cdots$, hence $X$ is noetherian.
  \par $(i) \Leftrightarrow (iii)$ and $(ii) \Leftrightarrow (iv)$ follow by
  taking complements.
\end{proof}
\begin{proof}[Proof of $(b)$]
  Let $\{X_i\}_{i \in I}$ be an open cover, and choose a well ordering on $I$.
  Then, consider the decreasing sequence of closed sets
  \begin{equation*}
    X \setminus X_1 \supseteq X \setminus (X_1 \cup X_2) \supseteq X \setminus (X_1 \cup X_2 \cup X_3) \supseteq \cdots.
  \end{equation*}
  Then, since $X$ is noetherian there exists a finite integer $r$ such that
  \begin{equation*}
    X \setminus (X_1 \cup \cdots \cup X_r) = \bigcap_{i \in I} X \setminus (X_1 \cup \cdots \cup X_i) = X \setminus \bigcup_{i \in I} X_i = \emptyset,
  \end{equation*}
  hence $X_1,\ldots,X_r$ is a finite subcover of $\{X_i\}_{i \in I}$.
\end{proof}
\begin{proof}[Proof of $(c)$]
  Let $A_1 \supset A_2 \supset \cdots$ be a descending chain of closed sets
  in $Y \subset X$; writing $A_i = Y \cap B_i$ for closed subsets
  $B_i \subset X$, we have that $B_1 \supset B_2 \supset \cdots$ stabilizes
  since $X$ is noetherian, so $A_1 \supset A_2 \supset \cdots$ does as well.
\end{proof}
\begin{proof}[Proof of $(d)$]
  Let $X$ be noetherian and Hausdorff.
  Hausdorff implies every finite point set is closed \cite[Thm.~17.8]{Mun00},
  hence it suffices to show $X$ is finite.
  So suppose $X$ is infinite; let $\Sigma$ be the family of infinite closed
  subsets, which is nonempty since $X \in \Sigma$.
  $\Sigma$ has a minimal element $Z$ by property $(ii)$ from $(a)$.
  Let $p,q \in Z$; since $X$ is Hausdorff there exist $U,V$ disjoint that
  contain $p,q$ respectively.
  We then have $X = (X \setminus U) \cup (X \setminus V)$, and so
  $Z = (Z \cap (X \setminus U) \cup Z\cap(X \setminus V))$.
  Then each of $Z \cap (X \setminus U)$ and $Z \cap (X \setminus V)$ are
  closed and strictly contained in $Z$, hence are finite by minimality of $Z$.
  This implies $Z$ is also finite, a contradiction.
\end{proof}

\begin{problem}\label{exc:1.8}
  Let $Y$ be an affine variety of dimension $r$ in $\AA^n$.
  Let $H$ be a hypersurface in $\AA^n$, and assume $Y \not\subseteq H$.
  Then every irreducible component of $Y \cap H$ has dimension $r-1$.
\end{problem}
\begin{proof}
  Let $A = k[x_1,\ldots,x_n]$.
  By Cor.~1.4, an irreducible component of $Y \cap H$ corresponds to a minimal prime
  in $A(Y \cap H)$, and so its dimension is equal to the coheight of the minimal
  prime corresponding to it by Thm.~1.8A$(b)$.
  Now $Y \cap H = Z(I(Y) + (f))$ by the proof of Prop.~1.1, so $I(Y \cap H) =
  \sqrt{I(Y) + (f)}$ by Prop.~1.2.
  By \cite[Prop.~1.1]{AM69}, the minimal primes in $A/\sqrt{I(Y) + (f)}$
  correspond to the minimal primes in $A$ containing $\sqrt{I(Y) + (f)}$; by
  \cite[Prop.~1.14]{AM69} these are exactly those minimal primes containing
  $I(Y) + (f)$.
  By \cite[Prop.~1.1]{AM69} again, these minimal primes are the minimal primes
  in $A(Y) = A/I(Y)$ containing $f$, and by Krull's Hauptidealsatz (Thm.~1.11A),
  they have height $1$, hence coheight $r-1$ in $A(Y)$.
  By the reductions above, we therefore have that the dimension of every
  irreducible component of $Y \cap H$ is $r-1$.
\end{proof}

\begin{problem}
  Let $\mathfrak{a} \subseteq A = k[x_1,\ldots,x_n]$ be an ideal which can be
  generated by $r$ elements.
  Then every irreducible component of $Z(\mathfrak{a})$ has dimension $\ge n-r$.
\end{problem}
\begin{proof}
  We induce on the number of generators $r$.
  Let $f_1,\ldots,f_r$ be our generators.
  The case $r=1$ is Prop.~$1.13$.
  So suppose the claim holds for $Y = Z(f_1,\ldots,f_{r-1})$; let $H = Z(f_r)$.
  By the proof of Prop.~1.1, $Y \cap H = Z(\mathfrak{a})$.
  Let $Y_i$ be the irreducible components of $Y$; by inductive hypothesis they
  have dimension $\ge n-r+1$.
  If $Y_i \subset H$, then $Y_i \cap H = Y_i$ is an irreducible component of
  $Z(\mathfrak{a})$ and has dimension $\ge n-r+1$.
  If $Y_i \not\subset H$, then each irreducible component of $Y_i \cap H$ has
  dimension $\ge n-r$ by Exercise \ref{exc:1.8}.
  In either case we have the irreducible components of $Z(\mathfrak{a})$ having
  dimension $\ge n-r$, so we are done.
\end{proof}

\begin{problem} \mbox{}
  \begin{enuma}
    \item If $Y$ is any subset of a topological space $X$ then $\dim Y \leq \dim X$. 
    \item If $X$ is a topological space which is covered by a family of open
      subsets $\{U_{i}\}$, then $\dim X = \sup \dim U_i$.
    \item Give an example of a topological space $X$ and a dense open subset $U$
      with $\dim U < \dim X$. 
    \item If Y is a closed subset of an irreducible finite-dimensional topological space $X$, and if $\dim Y = \dim X$, then $Y= X$. 
    \item Give an example of a noetherian topological space of infinite dimension.  
  \end{enuma}
\end{problem}
\begin{proof}[Proof of $(a)$]
  Let $\dim Y = n$ and $Y_0 \subsetneq Y_1 \subsetneq \cdots \subsetneq Y_n \subset
  Y$ be a chain of distinct irreducible closed subsets of $Y$. Take the closure of
  each $Y_i$. By Exercise \ref{exc:1.6}, each $\overline{Y_i}$ is irreducible.
  Hence, we see that 
  $\overline{Y_0} \subsetneq \overline{Y_1} \subsetneq \cdots \subsetneq
  \overline{Y_n}$ is an ascending chain of distinct irreducible closed subsets
  of $X$ ($\overline{Y_i} \ne \overline{Y_{i+1}}$ since if the two are equal,
  $Y_i = \overline{Y_i} \cap Y = \overline{Y_{i+1}} \cap Y = Y_{i+1}$,
  which is a contradiction).
  Hence $\dim X \ge n = \dim Y$. 
\end{proof}

\begin{proof}[Proof of $(b)$]
  Let $\dim X = n$. 
  From (a), $\dim U_i \le \dim X$. Hence, $\sup \dim U_i \le \dim X$.
  Now, consider an ascending chain $X_0 \subsetneq X_1 \subsetneq \cdots
  \subsetneq X_n$ of distinct irreducible closed subsets of $X$.
  Choose an open set $U_i$ such that $X_0 \cap U_i \neq \emptyset$.
  We can choose such $U_i$ because $\{U_i\}$ covers $X$. 
	Then, $U_i \cap X_j \neq \emptyset$ because $X_j \supset X_0$.
  Hence, $X_0 \cap U_i \subsetneq \cdots \subsetneq X_n \cap U_i$ is an
  ascending chain of irreducible closed subsets (with respect to the subspace
  topology) of $U_i$. Hence, $\sup \dim U_i \geq \dim X$.
\end{proof}
\begin{proof}[Proof of $(c)$]
	Let $X = {0,1}$, and $\tau = \{ \emptyset, X, \{0\}\}$ be the set of open sets of $X$. This is a topology. 
	Now, $\overline{\{0\}} = X$ so $U = {0}$ is dense in $X$. Note that ${0},{1}, {1,2}$ are all irreducible sets as it cannot be written as the union of two proper closed subsets with respect to the subspace topology.
	The dimension of $U$ is 0 as the only irreducible closed set (with respect to the subspace topology) of $U$ is $U$ itself. However, as ${1}$ is irreducible, we have an ascending chain of closed irreducible subsets of $X$ with length $1$, which is ${1} \subsetneq {0,1}$.  
\end{proof}

\begin{proof}[Proof of $(d)$]
	Let Y be a closed subset of an irreducible finite-dimensional topological space $X$ with $\dim Y = \dim X$. 
	Assume that $Y \subsetneq X$. Let $Y_0 \subsetneq \cdots \subsetneq Y_n \subset Y$ be the ascending chain of irreducible closed sets. Then, $Y_i$ is irreducible closed subset of $X$ as Y is closed. Because $X$ is also irreducible we have $Y_0 \subsetneq \cdots \subsetneq Y_n \subsetneq X$, so $\dim X geq n+1 > \dim Y$, which is a contradiction. Hence $X = Y$. 

\end{proof}

\begin{proof}[Proof of $(e)$]

	Let $X = \mathbb{N}$. Let $\tau = \{ \emptyset, X, \{2,3,\cdots\},
  \{3,4,\cdots\}, \cdots\}$. 
	
	$(X, \tau)$ is a topological space. Let $Y_1 \supset Y_2 \subset \cdots$ be any sequence of closed subsets. $X$ is noetherian because there exists an integer $r$ such that $Y_r = Y_{r+1} = \cdots$ by well-ordering principle. 
	
	Now, we show that the dimension of $X$ is infinite. We note that any closed set $\{n, n+1, \cdots\}$ is irreducible, so we have an infinite chain of closed irreducible subsets of $X$: $X \supsetneq \{2,3,4,\cdots\}\supsetneq \{3,4,5,\cdots\} \cdots $. Hence $\dim X = \infty$. 

\end{proof}

\printbibliography
\end{document}
