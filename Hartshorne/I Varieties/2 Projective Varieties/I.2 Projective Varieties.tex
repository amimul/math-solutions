\documentclass[12pt,letterpaper]{article}
\usepackage{geometry}
\geometry{letterpaper}
\usepackage{amsmath,amssymb,amsthm,mathrsfs}
\usepackage{mathtools}
\usepackage{ifpdf}
  \ifpdf
    \setlength{\pdfpagewidth}{8.5in}
    \setlength{\pdfpageheight}{11in}
  \else
\fi
\usepackage{hyperref}

\usepackage{tikz}
\usepackage{tikz-cd}
\usetikzlibrary{decorations.markings}
\tikzset{
  open/.style = {decoration = {markings, mark = at position 0.5 with { \node[transform shape] {\tikz\draw[fill=white] (0,0) circle (.3ex);}; } }, postaction = {decorate} },
  closed/.style = {decoration = {markings, mark = at position 0.5 with { \node[transform shape, xscale = .8, yscale=.4] {\upshape{/}}; } }, postaction = {decorate} },
  imm/.style = {decoration = {markings, mark = at position 0.3 with { \node[transform shape, xscale = .8, yscale=.4] {\upshape{/}}; }, mark = at position 0.6 with { \node[transform shape] {\tikz\draw[fill=white] (0,0) circle (.3ex);}; } }, postaction = {decorate} }
}

\usepackage{braket}

\usepackage{csquotes}
\usepackage[american]{babel}
\usepackage[style=alphabetic,firstinits=true,backend=biber,texencoding=utf8,bibencoding=utf8]{biblatex}
\bibliography{../../../References}
\AtEveryBibitem{\clearfield{url}}
\AtEveryBibitem{\clearfield{doi}}
\AtEveryBibitem{\clearfield{issn}}
\AtEveryBibitem{\clearfield{isbn}}
\renewbibmacro{in:}{}
\DeclareFieldFormat{postnote}{#1}
\DeclareFieldFormat{multipostnote}{#1}

\renewcommand{\theenumi}{$(\alph{enumi})$}
\renewcommand{\labelenumi}{\theenumi}

\newcounter{enumacounter}
\newenvironment{enuma}
{\begin{list}{$(\alph{enumacounter})$}{\usecounter{enumacounter} \parsep=0em \itemsep=0em \leftmargin=2.75em \labelwidth=1.5em \topsep=0em}}
{\end{list}}
\newcounter{enumicounter}
\newenvironment{enumi}
{\begin{list}{$(\roman{enumicounter})$}{\usecounter{enumicounter} \parsep=0em
\itemsep=0em \leftmargin=2.5em \labelwidth=1.75em \topsep=0em}}
{\end{list}}
\newcounter{enumdcounter}
\newenvironment{enumd}
{\begin{list}{$(\arabic{enumdcounter})$}{\usecounter{enumdcounter} \parsep=0em \itemsep=0em \leftmargin=1.75em \labelwidth=1.5em \topsep=0em}}
{\end{list}}
\newtheorem*{theorem}{Theorem}
\newtheorem*{universalproperty}{Universal Property}
\newtheorem{problem}{Exercise}[section]
\newtheorem{subproblem}{Problem}[problem]
\newtheorem{lemma}{Lemma}%[section]
\newtheorem{proposition}{Proposition}
\newtheorem{property}{Property}[problem]
\newtheorem*{lemma*}{Lemma}
\theoremstyle{definition}
\newtheorem*{definition}{Definition}
\newtheorem*{claim}{Claim}
\theoremstyle{remark}
\newtheorem*{remark}{Remark}

\numberwithin{equation}{section}
\numberwithin{figure}{problem}
\renewcommand{\theequation}{\arabic{section}.\arabic{equation}}

\DeclareMathOperator{\Ann}{Ann}
\DeclareMathOperator{\Ass}{Ass}
\DeclareMathOperator{\Supp}{Supp}
\DeclareMathOperator{\WeakAss}{\widetilde{Ass}}
\let\Im\relax
\DeclareMathOperator{\Im}{im}
\DeclareMathOperator{\Spec}{Spec}
\DeclareMathOperator{\SPEC}{\mathbf{Spec}}
\DeclareMathOperator{\Sp}{sp}
\DeclareMathOperator{\maxSpec}{maxSpec}
\DeclareMathOperator{\Hom}{Hom}
\DeclareMathOperator{\Soc}{Soc}
\DeclareMathOperator{\Ht}{ht}
\let\AA\relax
\DeclareMathOperator{\AA}{\mathbb{A}}
\DeclareMathOperator{\V}{\mathbf{V}}
\DeclareMathOperator{\Aut}{Aut}
\DeclareMathOperator{\Char}{char}
\DeclareMathOperator{\Frac}{Frac}
\DeclareMathOperator{\Proj}{Proj}
\DeclareMathOperator{\stimes}{\text{\footnotesize\textcircled{s}}}
\DeclareMathOperator{\End}{End}
\DeclareMathOperator{\Ker}{Ker}
\DeclareMathOperator{\Coker}{coker}
\DeclareMathOperator{\LCM}{LCM}
\DeclareMathOperator{\Div}{Div}
\DeclareMathOperator{\id}{id}
\DeclareMathOperator{\Cl}{Cl}
\DeclareMathOperator{\dv}{div}
\DeclareMathOperator{\Gr}{Gr}
\DeclareMathOperator{\pr}{pr}
\DeclareMathOperator{\trdeg}{trdeg}
\DeclareMathOperator{\rank}{rank}
\DeclareMathOperator{\codim}{codim}
\DeclareMathOperator{\sgn}{sgn}
\DeclareMathOperator{\GL}{GL}
\newcommand{\GR}{\mathbb{G}\mathrm{r}}
\newcommand{\gR}{\mathrm{Gr}}
\newcommand{\EE}{\mathscr{E}}
\newcommand{\FF}{\mathscr{F}}
\newcommand{\GG}{\mathscr{G}}
\newcommand{\HH}{\mathscr{H}}
\newcommand{\II}{\mathscr{I}}
\newcommand{\LL}{\mathscr{L}}
\newcommand{\MM}{\mathscr{M}}
\newcommand{\OO}{\mathcal{O}}
\newcommand{\Ss}{\mathscr{S}}
\newcommand{\Af}{\mathfrak{A}}
\newcommand{\Aa}{\mathscr{A}}
\newcommand{\PP}{\mathbb{P}}
\newcommand{\red}{\mathrm{red}}
\newcommand{\Sh}{\mathfrak{Sh}}
\newcommand{\Psh}{\mathfrak{Psh}}
\newcommand{\LRS}{\mathsf{LRS}}
\newcommand{\Sch}{\mathfrak{Sch}}
\newcommand{\Var}{\mathfrak{Var}}
\newcommand{\Rings}{\mathfrak{Rings}}
\DeclareMathOperator{\In}{in}
\DeclareMathOperator{\Ext}{Ext}
\DeclareMathOperator{\Spe}{Sp\acute{e}}
\DeclareMathOperator{\HHom}{\mathscr{H}\!\mathit{om}}
\newcommand{\isoto}{\overset{\sim}{\to}}
\newcommand{\isolongto}{\overset{\sim}{\longrightarrow}}
\newcommand{\Mod}{\mathsf{mod}\mathchar`-}
\newcommand{\MOD}{\mathsf{Mod}\mathchar`-}
\newcommand{\gr}{\mathsf{gr}\mathchar`-}
\newcommand{\qgr}{\mathsf{qgr}\mathchar`-}
\newcommand{\uqgr}{\underline{\mathsf{qgr}}\mathchar`-}
\newcommand{\qcoh}{\mathsf{qcoh}\mathchar`-}
\newcommand{\Alg}{\mathsf{Alg}\mathchar`-}
\newcommand{\coh}{\mathsf{coh}\mathchar`-}
\newcommand{\vect}{\mathsf{vect}\mathchar`-}
\newcommand{\imm}[1][imm]{\hspace{0.75ex}\raisebox{0.58ex}{%
\begin{tikzpicture}[commutative diagrams/every diagram]
\draw[commutative diagrams/.cd, every arrow, every label,hook,{#1}] (0,0ex) -- (2.25ex,0ex);
\end{tikzpicture}}\hspace{0.75ex}}
\newcommand{\dashto}[2]{\smash{\hspace{-0.7em}\begin{tikzcd}[column sep=small,ampersand replacement=\&] {#1} \rar[dashed] \& {#2} \end{tikzcd}\hspace{-0.7em}}}

\usepackage{todonotes}

\title{Hartshorne Ch.~I, \S2 Projective Varieties}
\author{Takumi Murayama, Kyu Jun}

\begin{document}
\maketitle
\setcounter{section}{2}
\begin{problem}\label{exc:2.1}
  Prove the ``homogeneous Nullstellensatz,'' which says if $\mathfrak{a}
  \subseteq S$ is a homogeneous ideal, and if $f \in S$ is a homogeneous
  polynomial with $\deg f > 0$, such that $f(P) = 0$ for all $P \in
  Z(\mathfrak{a})$ in $\PP^n$, then $f^q \in \mathfrak{a}$ for some $q >
  0$.
\end{problem}
\begin{proof}
  Consider $S = k[x_0,\ldots,x_n]$ as an affine polynomial ring, and let
  $CZ(\mathfrak{a})$ be the affine variety defined by $\mathfrak{a}$ in
  $\AA^{n+1}$.
  Then, $f(P) = 0$ for any $P \in CZ(\mathfrak{a})$, since any nonzero $P$ is a
  representative of some point in $Z(\mathfrak{a})$, and if $P=0$ then $f(P) =
  0$ by the fact that $f$ is homogeneous with $\deg f > 0$.
  By the affine Nullstellensatz (Thm.~1.3A), we have $f^q \in \mathfrak{a}$ for
  some $q > 0$.
\end{proof}

\begin{problem}
  For a homogeneous ideal $\mathfrak{a} \subseteq S$, show that the following
  conditions are equivalent:
  \begin{enumi}
    \item $Z(\mathfrak{a}) = \emptyset$ (the empty set);
    \item $\sqrt{\mathfrak{a}} =$ either $S$ or the ideal $S_+ =
      \bigoplus_{d > 0}S_d$;
    \item $\mathfrak{a} \supseteq S_d$ for some $d > 0$. 
  \end{enumi}
\end{problem}
\begin{proof}
  $(i) \Rightarrow (ii)$. If $Z(\mathfrak{a}) = \emptyset$, then
  $CZ(\mathfrak{a}) = \{0\}$ or $\emptyset$, where as in Exercise \ref{exc:2.1}
  $CZ(\mathfrak{a})$ is the affine variety in $\AA^{n+1}$ defined by
  $\mathfrak{a}$.
  If $CZ(\mathfrak{a}) = \{0\}$, then $x_i^r \in \mathfrak{a}$ by the affine
  Nullstellensatz (Thm.~1.3A) for each $0 \le i \le n$, hence $x_i \in
  \sqrt{\mathfrak{a}}$ for each $0 \le i \le n$, and so $\sqrt{\mathfrak{a}}
  \supset S_+$; this is moreover an equality since $S_0 = k$, and $\mathfrak{a}
  \cap k = \emptyset$ for otherwise $CZ(\mathfrak{a}) = \emptyset$.
  On the other hand, if $CZ(\mathfrak{a}) = \emptyset$, then since $\emptyset =
  Z(S)$ by the proof of Prop.~1.1, then $I(\emptyset) = I(CZ(\mathfrak{a})) =
  \sqrt{\mathfrak{a}} = S$ by Prop.~$1.2(d)$.
  \par $(ii) \Rightarrow (iii)$. If $\sqrt{\mathfrak{a}} = S$, then $1 \in
  \sqrt{\mathfrak{a}}$, which implies $1 = 1^n \in \mathfrak{a}$, and so $\mathfrak{a} \supset S_d$ for any $d$.
  On the other hand, if $\sqrt{\mathfrak{a}} = S_+$, then $x_i^r \in
  \mathfrak{a}$ for some $r$ for all $0 \le i \le n$. But every element in
  $S_{r(n+1)}$ has some $x_i^r$ appearing in each term by the pigeon-hole
  principle, and so $\mathfrak{a} \supset S_{r(n+1)}$.
  \par $(iii) \Rightarrow (i)$. $\mathfrak{a} \supset S_d$ implies all $x_i^d$
  must vanish on $Z(\mathfrak{a})$, which is impossible in $\PP^n$, hence
  $Z(\mathfrak{a}) = \emptyset$.
\end{proof}

\begin{problem}\mbox{}
  \begin{enuma}
    \item If $T_1 \subseteq T_2$ are subsets of $S^h$, then $Z(T_1) \supseteq
      Z(T_2)$.
    \item If $Y_1 \subseteq Y_2$ are subsets of $\PP^n$, then $I(Y_1) \supseteq
      I(Y_2)$.
    \item For any two subsets $Y_1,Y_2$ of $\PP^n$, $I(Y_1 \cup Y_2) = I(Y_1)
      \cap I(Y_2)$.
    \item If $\mathfrak{a} \subseteq S$ is a homogeneous ideal with
      $Z(\mathfrak{a}) \ne \emptyset$, then $I(Z(\mathfrak{a})) =
      \sqrt{\mathfrak{a}}$.
    \item For any subset $Y \subseteq \PP^n$, $Z(I(Y)) = \overline{Y}$.
  \end{enuma}
\end{problem}
\begin{proof}[Proof of $(a)$]
  For $i = 1,2$, let $CZ(T_i)$ be the affine variety defined by $T_i$ in
  $\AA^{n+1}$. Then, from Prop.~1.2, $CZ(T_1)\supset CZ(T_2)$. $Z(T_i)$ is the projection of $CZ(T_i)$ to $\PP^n$, so it follows that $Z(T_1) \supset Z(T_2)$.  
\end{proof}
\begin{proof}[Proof of $(b)$--$(d)$]
  Obvious. Do pretty much the same thing as in prop 1.2 (or use affine cones or
  homogeneous ideals in $k[x_0,\cdots,x_n]$ if applicable).
\end{proof}
\begin{proof}[Proof of $(e)$]
  Because $Z(I(Y))$ is closed, by the definition of the closure, $Z(I(Y))
  \supset Y$ implies $Z(I(Y)) \supset \overline{Y}$. Now let $W$ be any closed set containing $Y$. Then $W = Z(\mathfrak{a})$ for some homogeneous ideal $\mathfrak{a}$. Then from (b), (d) and the fact that an ideal is contained in its radical ideal, $\mathfrak{a} \subseteq \sqrt{\mathfrak{a}}= IZ(\mathfrak{a}) \subseteq I(Y)$. Hence, $\mathfrak{a} \subseteq I(Y)$ So from (a), we have $W = Z(\mathfrak{a}) \supset ZI(Y)$. Hence, $ZI(Y) = \overline{Y}$
\end{proof}

\begin{problem}\mbox{}
  \begin{enuma}
    \item There is a one-to-one inclusion-reversing correspondence between algebraic sets in $\PP^n$ and homogeneous radical ideals of $S$ not equal to $S_+$ given by $Y \mapsto I(Y)$ and $\mathfrak{a} \mapsto Z(\mathfrak{a})$
    \item An algebraic set $Y \subseteq \PP^n$ is irreducible if and only if $I(Y)$ is a prime ideal. 
    \item Show that $\PP^n$ itself is irreducible. 
   \end{enuma}
\end{problem}
\begin{proof}\mbox{}
  \par (a) This follows from exercise 2.2, 2.3. 
  \par (b) Let $Y= Z(T)$ for a set of homogeneous polynomials of $k[x_0, \cdots, x_n]$. Let $CY \subset \AA^{n+1}$ be the affine variety defined by $T$. Then, as the projection map $\AA^{n+1} \to (\AA^{n+1}-\{(0,\cdots,0)\})/k^{\times} \simeq \PP^{n}$ is surjective and continuous, $Y$ is irreducible $\iff CY$ is irreducible. From cor 1.4, $CY$ is irreducible if and only if $I(CY)$ is prime. $ I(CY)$ is prime iff $I(Y)$ is prime. Hence, $Y$ is irreducible if and only if $I(Y)$ is prime. 
  \par (c) There are several ways to show $\PP^n$ is irreducible. One way you can do is to use that $U_0 \simeq \AA^n$ is irreducible, and $\PP^n = \overline{U_0}$ is irreducible as it is a closure of an irreducible subset (example 1.14). 
  One another way is to see that $\PP^n = Z(\mathfrak{o})$, where $\mathfrak{o} = (0)$. As $(0)$ is a homogeneous prime ideal, $\PP^n$ is irreducible (from (b).  
\end{proof}

\begin{problem} \mbox{}
  \begin{enuma}
    \item $\PP^n$ is a noetherian topological space. 
    \item Every algebraic set in $\PP^n$ can be written uniquely as a finite union of irreducible algebraic sets, no one containing another. These are called its irreducible components. 
  \end{enuma}
\end{problem}
\begin{proof}
  \par (a) Let $Y_1 \supseteq Y_2 \supset Y_3 \supset \cdots$ is a descending chain of closed subsets of $\PP^n$, then $I(Y_1) \subseteq I(Y_2) \subseteq \cdots$ is an ascending chain of homogeneous radical ideals in $S = k[x_0, \cdots, x_n]$ (from 2.4 (a)). Then, this chain of ideals eventually stabilizes as $S$ is a noetherian ring. As for each $i$, $Y_i = ZI(Y_i)$, this means that the descending chain of $Y_i$ also eventually stabilizes.
  \par (b) From (a), $\PP^n$ is noetherian, so by prop 1.5, every algebraic set in $\PP^n$ can be written uniquely as a finite union of irreducible algebraic sets, no one containing another. 
\end{proof}

\begin{problem} If $Y$ is a projective variety with homogeneous coordinate ring $S(Y)$, show that $\dim S(Y) = \dim Y +1$. 
\end{problem}
\begin{proof}
  Let $\phi_i: U_i \to \AA^n$ be the homeomorphism of prop 2.2, and $Y_i = \phi(Y \cap U_i)$. Then $A(Y_i)$ is its affine coordinate ring. Choose $i$ such that $\dim Y_i = \sup_{j} \dim Y_j = \dim Y$ (problem 1.10 (b)). 
  \par Then, any element $\frac{f}{x_i^n} \in S(Y)_{x_i}$ of degree $0$ can be expressed as $$f(\frac{x_0}{x_i}, \cdots, \frac{x_{i-1}}{x_i}, 1, \frac{x_{i+1}}{x_i}, \cdots, \frac{x_n}{x_i})$$, which is $\alpha(f) \in A(Y_i)$ where $\alpha$ is a function defined in prop 2.2. 
  Conversely, for $f \in A(Y_i)$ of degree $e$, consider a function $\beta: A(Y_i) \to (S(Y)_{x_i})_0$ such that $f \to {x_i}^{-e}f$, where $(S(Y)_{x_i})_0$, is the subring of elements of degree $0$ of the localized ring $S(Y)_{x_i}$. Then $\alpha \beta$, $\beta \alpha$ are identity maps. Thus, $A(Y_i)$ can be identified with $(S(Y)_{x_i})_0$. Then, $S(Y)_{x_i} = ((S(Y)_{x_i})_0)[x_i, x_i^{-1}] \simeq A(Y_i)[x_i, x_i^{-1}]$
  \par From thm 1.8A, the transcendence degree of $K(A(Y_i)[x_i, x_i^{-1}]$ is one plus the transcendence degree of $K(A(Y_i))$. From prop 1.7, the transcendence degree of $K(A(Y_i)) = \dim(Y_i)$. Hence,  
  $$\dim S(Y) = \dim S(Y)_{x_i} = \dim Y_i +1 = \dim Y+1$$
  $\dim Y_i = \dim Y$ whenever $Y_i \neq \emptyset$ because if $Y_i$ is dense in $Y$ if $Y_i \neq \emptyset$ (from problem 2.7(b) which we will soon show)
\end{proof}

\begin{problem}\mbox{}
  \begin{enuma}
    \item $\dim \PP^n = n$
    \item If $Y \subseteq \PP^n$ is a quasi-projective variety, then $\dim Y = \dim \overline{Y}$
  \end{enuma}
\end{problem}

\begin{proof}
  \par (a) from exercise 2.6, $\dim S(\PP^n) = n + 1 = \dim(\PP^n) +1$. Hence, $\dim \PP^n = n$
  \par (b) From exercise 2.6, $\dim S(Y) = \dim Y +1 \iff \dim Y = \dim S(Y) -1 $. We note that $\dim S(Y) = \dim (CY)$, where $CY = \pi^{-1} (Y)$. $\pi : \AA^{n+1} \setminus \{0\} \to \PP^n$. This is true because $S(Y) = k[x_0, \cdots, x_n]/I(Y)$, and realizing this in an affine space $\AA^{n+1}$, we get $\dim S(Y) = \dim CY$(from prop 1.7). Then, from prop 1.10, $\dim S(Y) = \dim CY = \dim \overline{CY} = \dim C \overline{Y} = \dim S(\overline{Y})$. Hence, $\dim Y = \dim S(Y) -1 = \dim S(\overline{Y}) -1 = \dim \overline{Y}$.
\end{proof}

\begin{problem} A projective variety $Y \subseteq \PP^n$ has dimension $n-1$ if and only if it is the zero set of a single irreducible homogeneous polynomial $f$ of positive degree. $Y$ is called a hypersurface in $\PP^n$. 
\end{problem}
\begin{proof}
  Let $Y$ be a projective variety of dimension $n-1$. Then $\dim S(Y) = n$ from exercise 2.6. Regarded as a polynomial in $k[x_0,\cdots,x_n]$ (i.e. considering $Z( S(Y))$ an affine space $\AA^{n+1}$), $CY = Z(S(Y))$ is a $n$-dimensional variety. Using prop 1.13, an affine cone $CY$ of $Y$ is the zero set $Z(f)$ of a single nonconstant irreducible polynomial. Thus, $Y = Z(f')$ where $f'$ the homogenization of $f$. 
  \par Conversely, let $Y= Z(F)$ where $f \in  S = k[x_0, \cdots, x_n]$ be an irreducible homogeneous polynomial of positive degree (thus $f \neq 0$). By Hauptidealsatz (theorem 1.11A), $(f)$ has height 1, as $(f)$ is a minimal prime ideal containing $f$. Hence, in an affine cone, $CY = Z(f)$ has dimension $n+1-1 =n$ (by theorem 1.8A). Hence, $\dim Y = \dim CY-1 = n-1$. 
  \par cf. It is quite obvious (to Takumi who is quite unhappy to see the phrase "quite obvious": I know you are unhappy, but I-perhaps you- will show this soon! :p) that $\dim Y = \dim CY -1$, and we will show that in exercise 2.10 (c). 
\end{proof}

\begin{problem} \textbf{Projective Closure of an Affine Variety} If $Y \subseteq \AA^n$ is an affine variety, we identify $\AA^n$ with an open set $U_0 \subset \PP^n$ by the homeomorphism $\phi_0$. Then we can speak of $\overline{Y}$, the closure of $Y$ in $\PP^n$, which is called the projective closure of $Y$. 
 \begin{enuma}
   \item Show that $I(\overline{Y})$ is the ideal generated by $\beta(I(Y))$, using the notation of the proof of (2.2). 
   \item Let $Y \subset \AA^n$ be the twisted cubic of (Ex 1.2). Its projective closure $\overline{Y} \subset \PP^n$ is called the twisted cubic curve in $\PP^3$. Find generators for $I(Y)$ and $I(\overline{Y})$, and use this example to show that if $f_1, \cdots, f_r$ generate $I(Y)$, then $\beta(f_1), \cdots, \beta(f_r)$ do not necessarily generate $I(\overline{Y})$. 
 \end{enuma}
\end{problem}

\begin{proof} [proof of (a)]
  Let $F \in I(\overline{Y})$. Then, $f =\phi_0(F) = F(1,x_1, \cdots, x_n) \in I(Y)$ as $F(1,y) = 0$ for all $y \in Y$. As $\beta(f) = F$, we conclude that $F \in \beta(I(Y))$. Hence, $I(\overline{Y}) \subseteq \beta(I(Y))$. 
  \par Conversely, let $F \in (\beta(I(Y))$. Then, $F = \sum_i a_i \beta(f_i)$, where each $f_i \in I(Y)$. Then, $\beta(f_i) \in I(\overline{Y})$, as $\beta(f_i)[1:y] = 0$ for all $y \in Y$. Hence, $F \in I(\overline{Y})$. 
\end{proof}

\begin{proof} [proof of (a)]
  From exercise 1.2, $I(Y) = (y-x^2, z-x^3)$. Now, Using $\phi_0$, $Y$ can be realized as a subset of $\PP^3= \{[w:x:y:z]\}$ such that $Y = \{[1:t:t^2:t^3]: t \in k \} \subset \PP^3$. Homogenizing the coordinates, we get $\overline{Y} = \{[s^3:s^2t:st^2:t^3]| s,t \in \AA, \mbox{ either } s \mbox{ or } t \mbox{ is nonzero}\} \subset \PP^3$
  \par Thus, $I(\overline{Y}) = (wz - xy, wy -x^2, w^2z-x^4) \neq (\beta(z-x^3), \beta(y-x^2)$. Thus, $\beta(f_1), \cdots, \beta(f_r)$ do not necessarily generate $I(\overline{Y})$ even if $f_1, \cdots, f_r$ generate $I(Y)$.
\end{proof}

\begin{problem} \textbf{The Cone Over a Projective Variety} Let $Y \subset \PP^n$ be a nonempty algebraic set, and let $\theta : \AA^{n+1} \setminus \{(0,\cdots,0\} \to \PP^n$ be the map which sends the point with affine coordinates $(a_0, \cdots, a_n)$ to the point with homogeneous coordinates $[a_0 :\cdots: a_n]$. We define the affine cone over $Y$ to be $$C(Y) = \theta^{-1}(Y) \cup \{(0,\cdots, 0)\}.$$ 
  \begin{enuma}
    \item Show that $C(Y)$ is an algebraic set in $\AA^{n+1}$, whose ideal is equal to $I(Y)$, considered as an ordinary ideal in $k[x_0, \cdots, x_n]$. 
    \item $C(Y)$ is irreducible if and only if $Y$ is irreducible. 
    \item $\dim C(Y) = \dim Y +1$
  \end{enuma}
\end{problem}
\begin{proof}[proof of (a)]


\end{proof}

\printbibliography
\end{document}
