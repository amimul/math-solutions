\documentclass[12pt,letterpaper]{article}
\usepackage{geometry}
\geometry{letterpaper}
\usepackage{amsmath,amssymb,amsthm,mathrsfs}
\usepackage{mathtools}
\usepackage{ifpdf}
  \ifpdf
    \setlength{\pdfpagewidth}{8.5in}
    \setlength{\pdfpageheight}{11in}
  \else
\fi
\usepackage{hyperref}

\usepackage[utf8]{inputenc}
\usepackage{csquotes}
\usepackage[american]{babel}
%\usepackage[style=alphabetic,firstinits=true,backend=biber,texencoding=utf8,bibencoding=utf8]{biblatex}
%\bibliography{../../References}
%\AtEveryBibitem{\clearfield{url}}
%\AtEveryBibitem{\clearfield{doi}}
%\AtEveryBibitem{\clearfield{issn}}
%\AtEveryBibitem{\clearfield{isbn}}
%\renewbibmacro{in:}{}
%\DeclareFieldFormat{postnote}{#1}
%\DeclareFieldFormat{multipostnote}{#1}

\newtheorem{problem}{Problem}[section]
\renewcommand{\theproblem}{\thesection-\Alph{problem}}
\newtheorem{lemma}{Lemma}%[section]

\numberwithin{equation}{section}

%Put Macros here
\newcommand{\RR}{\ensuremath \mathbf{R}}

\title{Solutions to Milnor's \emph{Characteristic Classes}}
\author{Takumi Murayama, Matt Stevenson, Feng Zhu}

\begin{document}
\maketitle
\section{Smooth Manifolds}
\begin{problem}
  Let $M_1 \subset \RR^A$ and $M_2 \subset \RR^B$ be smooth manifolds. Show that
  $M_1 \times M_2 \subset \RR^A \times \RR^B$ is a smooth manifold, and that the
  tangent manifold $D(M_1 \times M_2)$ is canonically diffeomorphic to the
  product $DM_1 \times DM_2$. Note that a function $x \mapsto (f_1(x),f_2(x))$
  from $M$ to $M_1 \times M_2$ is smooth if and only if both $f_1 \colon M \to
  M_1$ and $f_2 \colon M \to M_2$ are smooth.
\end{problem}
\begin{proof}
  \textbf{Proof would go here}.
\end{proof}

\section{Vector Bundles}

\begin{problem}
Show that the unit sphere $S^n$ admits a vector field which is nowhere zero, providing that $n$ is odd. Show that the normal bundle of $S^n \subset \RR^{n+1}$ is trivial for all $n$.
\end{problem}

\begin{proof}
Let $S^n \subset \RR^{n+1}$ be the standard embedding and let $(x^1,\ldots,x^{n+1})$ be the standard coordinates on $\RR^{n+1}$. For each $\vec{x} = (x^1,\ldots,x^{n+1}) \in S^n$, define $v(\vec{x}) := (x^2, -x^1, \ldots, x^{n+1}, -x^n) \in \RR^{n+1}$. This function $v \colon S^n \to TS^n$ is a vector field, since 
\[
v(\vec{x}) \cdot \vec{x}  = (x^1 x^2 - x^2 x^1 ) + \ldots + (x^{n} x^{n+1} - x^{n+1} x^n) = 0 \textrm{ for all $\vec{x} \in S^n$.}
\]
It is clear that $v$ does not vanish.

To each $\vec{x} \in S^n$, let $s(\vec{x}) = (\vec{x},\vec{x})$, where the first coordinate denotes the basepoint and the second coordinate denotes the point on the line through the origin and $\vec{x}$. As the normal bundle of $S^n \subset \RR^{n+1}$ has rank 1, the existence of the nowhere vanishing section $s$ implies that the bundle is trivial, by Theorem 2.2.

\end{proof}

\begin{problem}
If $S^n$ admits a vector field which is nowhere zero, show that the identity map of $S^n$ is homotopic to the antipodal map. For $n$ even, show that the antipodal map of $S^n$ is homotopic to the reflection 
\[
r(x^1,\ldots,x^{n+1}) = (-x^1,x^2,\ldots,x^{n+1}),
\]
and therefore has degree $-1$. Combining these facts, show that $S^n$ is not parallelizable for $n$ even, $n \geq 2$.
\end{problem}

\begin{proof}
Let $v \colon S^n \to \RR^{n+1}$ be a nonwhere zero vector field, then for $t \in [0,\pi]$,
\[
(\vec{x},t) \mapsto H(\vec{x},t) := \cos(t) \vec{x} + \sin(t) \frac{v(\vec{x})}{\| v(\vec{x}) \|^2 }
\]
defines a homotopy from the identity map to the antipodal map (remark that $v$ is needed so that the homotopy stays on $S^n$: indeed, for any fixed $t$, $H(\vec{x},t) \cdot H(\vec{x},t) = 1$ since $\vec{x} \cdot v(\vec{x}) = 0$.)

Middle proof still to be added.

Let $n \geq 2$ be even and assume that $S^n$ is parallelizable, then there exist $n$ nowhere dependent (hence, nowhere zero) sections of $S^n$. Thus, the identity map is homotopic to the antipodal map, which is in turn homotopic to the above reflection; taking degrees, we find that $1= \deg(id) = \deg(-id) = \deg(r) = -1$, a contradiction.
\end{proof}

\begin{problem}
(Existence theorem for Euclidean metrics) Using a partition of unity, show that any vector bundle over a paracompact base can be given a Euclidean metric. 
\end{problem}

\begin{proof}
Let $\{ U_{\alpha} \}$ be an open cover of the base space $M$ such that the vector bundle $(E,M,\pi)$ is trivial over each $U_{\alpha}$. Over each $U_{\alpha}$, we can equip $\pi^{-1}(U_{\alpha}) \simeq U_{\alpha} \times \RR^{\textrm{rank}(E)}$ with a Euclidean metric $\langle \cdot, \cdot \rangle_{\alpha}$. By paracompactness, there is a partition of unity $\{ \psi_{\alpha} \}$ subordinate to $\{ U_{\alpha} \}$, so $\langle \cdot , \cdot \rangle := \sum \psi_{\alpha} \langle \cdot , \cdot \rangle_{\alpha}$ is a Euclidean metric on $E$.
\end{proof}

\begin{problem}
The Alexandroff line $L$ (also called the ``long line'') is a smooth, connected, 1-dimensional manifold which is not paracompact. Show that $L$ cannot be given a Riemannian metric.
\end{problem}

\begin{proof}
A Riemannian metric induces a metric by taking the infimum of the lengths of paths connecting 2 points, but $L$ is not metrizable (proof needed).
\end{proof}
\end{document}
