\documentclass[12pt]{article}
\usepackage{geometry}
\geometry{letterpaper}
\usepackage[utf8]{inputenc}
\usepackage[unicode]{hyperref}
\usepackage{amsmath,amsthm,amssymb}
\usepackage{mathtools}
\usepackage{ifpdf}
  \ifpdf
    \setlength{\pdfpagewidth}{8.5in}
    \setlength{\pdfpageheight}{11in}
  \else
\fi

\usepackage{tikz}
\usetikzlibrary{decorations.markings,calc,shapes,snakes}
\usepackage{pgflibraryarrows}
\usepackage{tikz-cd}
\tikzset{->-/.style={decoration={
  markings,
  mark=at position #1 with {\arrow[scale=2,draw=black]{>}}},postaction={decorate}}}
\tikzset{-<-/.style={decoration={
  markings,
  mark=at position #1 with {\arrow[scale=2,draw=black]{<}}},postaction={decorate}}}
\tikzset{->--/.style={decoration={
  markings,
  mark=at position #1 with {\arrow[scale=1.2]{>}}},postaction={decorate}}}
\tikzset{label/.style={%
  postaction={ decorate,transform shape,
  decoration={ markings, mark=at position .55 with \node #1;}}}}
\tikzset{decorate sep/.style 2 args=
  {decorate,decoration={shape backgrounds,shape=circle,shape size=#1,shape sep=#2}}}

\usepackage{csquotes}
\usepackage[american]{babel}
\usepackage[style=alphabetic,firstinits=true,backend=biber,texencoding=utf8,bibencoding=utf8]{biblatex}
\bibliography{../References}
\AtEveryBibitem{\clearfield{url}}
\AtEveryBibitem{\clearfield{doi}}
\AtEveryBibitem{\clearfield{issn}}
\AtEveryBibitem{\clearfield{isbn}}
\renewbibmacro{in:}{}
\DeclareFieldFormat{postnote}{#1}
\DeclareFieldFormat{multipostnote}{#1}

\usepackage{bm}
\usepackage{bookmark}

\newtheorem{probaux}[subsubsection]{Exercise}
\newtheorem*{claim}{Claim}
\newtheorem*{lemma}{Lemma}%[subsubsection]
\theoremstyle{remark}
\newtheorem*{remark}{Remark}

%\renewcommand{\thesubsection}{\arabic{subsection}}
%\renewcommand{\thelemma}{\thesubsubsection\alph{lemma}}

\usepackage{xparse}
\NewDocumentEnvironment{problem}{o}
 {\IfNoValueTF{#1}
   {\probaux\addcontentsline{toc}{subsubsection}{\protect Exercise \thesubsubsection}}
   {\probaux[#1]\addcontentsline{toc}{subsubsection}{\protect Exercise \thesubsubsection}}%
   \ignorespaces}
 {\endprobaux}

\usepackage{shorttoc}
\usepackage[toc]{multitoc}
\usepackage{tocloft}

\newcounter{enumacounter}
\newenvironment{enuma}
{\begin{list}{$(\alph{enumacounter})$}{\usecounter{enumacounter} \parsep=0em \itemsep=0em \leftmargin=2.75em \labelwidth=1.5em \topsep=0em}}
{\end{list}}
\newcounter{enumicounter}
\newenvironment{enumi}
{\begin{list}{$(\roman{enumicounter})$}{\usecounter{enumicounter} \parsep=0em \itemsep=0em \leftmargin=2.25em \labelwidth=2em \topsep=0em}}
{\end{list}}

\DeclareMathOperator{\Aut}{Aut}
\let\Im\relax
\DeclareMathOperator{\Im}{im}
\DeclareMathOperator{\lcm}{lcm}
\DeclareMathOperator{\id}{id}
\DeclareMathOperator{\Hom}{Hom}
\newcommand{\RR}{\mathbb{R}}
\newcommand{\GL}{\mathit{GL}}
\newcommand{\PGL}{\mathit{PGL}}
\newcommand{\SL}{\mathit{SL}}
\newcommand{\bracket}[1]{[#1]}
\let\amsamp=&

\usepackage{braket}
\DeclareMathOperator*{\bigast}{\raisebox{-0.6ex}{\scalebox{2.5}{$\ast$}}}

\title{Selected Solutions to Hatcher's Algebraic Topology}
\author{Takumi Murayama}

\begin{document}
\maketitle
These solutions are the result of taking MAT560 Algebraic Topology in the Spring
of 2013 at Princeton University. This is not a \emph{complete} set of solutions; see the \hyperlink{det.1}{List of Solved Exercises} at the end. Please e-mail \href{mailto:takumim@umich.edu}{\nolinkurl{takumim@umich.edu}} with any corrections.
\pdfbookmark[1]{Contents}{toc}
\begingroup
\setlength{\cftsubsecnumwidth}{2.75em}
\shorttoc{Contents}{2}
\endgroup
\newpage
\setcounter{section}{-1}
\section{Some Underlying Geometric Notions}
\begingroup
\renewcommand{\thesubsubsection}{\thesection.\arabic{subsubsection}}
\setcounter{subsubsection}{0}
\begin{problem}
  Construct an explicit deformation retraction of the torus with one point deleted onto a graph consisting of two circles intersecting in a point, namely, longitude and meridian circles of the torus.
\end{problem}
\begin{proof}
  Using the CW complex construction of the torus on p.~5, we have the map denoted by the small arrows:
  \begin{center}
    \begin{tikzpicture}[line cap=round,line join=round,x=1.0cm,y=1.0cm]
      \node (a) at (0,0) {
        \begin{tikzpicture}[>=stealth]
          \fill [gray] (0,0) rectangle (2,2);
          \draw (0,1) node[anchor=east] {$a$};
          \draw (2,1) node[anchor=west] {$a$};
          \draw (1,0) node[anchor=north] {$b$};
          \draw (1,2) node[anchor=south] {$b$};
          \draw [->-=0.55] (0,0) -- (0,2);
          \draw [->-=0.55] (0,2) -- (2,2);
          \draw [->-=0.55] (2,0) -- (2,2);
          \draw [->-=0.55] (0,0) -- (2,0);
          \fill [white] (1,1) circle (0.1); 
          \draw (1,1) circle(0.1);
        \end{tikzpicture}
      };
      \node (b) at (a.east) [anchor=west,xshift=0.5cm] {
        \begin{tikzpicture}[>=stealth]
          \fill [gray] (0,0) rectangle (2,2);
          \draw (0,1) node[anchor=east] {$a$};
          \draw (2,1) node[anchor=west] {$a$};
          \draw (1,0) node[anchor=north] {$b$};
          \draw (1,2) node[anchor=south] {$b$};
          \draw [->-=0.55] (0,0) -- (0,2);
          \draw [->-=0.55] (0,2) -- (2,2);
          \draw [->-=0.55] (2,0) -- (2,2);
          \draw [->-=0.55] (0,0) -- (2,0);

          \draw [->--=0.55] (1,1) -- (2,0);
          \draw [->--=0.55] (1,1) -- (2,2);
          \draw [->--=0.55] (1,1) -- (0,0);
          \draw [->--=0.55] (1,1) -- (0,2);
          \draw [->--=0.56] (1,1) -- (1,0);
          \draw [->--=0.56] (1,1) -- (1,2);
          \draw [->--=0.56] (1,1) -- (0,1);
          \draw [->--=0.56] (1,1) -- (2,1);

          \fill [white] (1,1) circle (0.1); 
          \draw (1,1) circle(0.1);
        \end{tikzpicture}

      };
      \node (c) at (b.east) [anchor=west,xshift=0.5cm] {
        \begin{tikzpicture}[>=stealth]
          \draw (0,1) node[anchor=east] {$a$};
          \draw (2,1) node[anchor=west] {$a$};
          \draw (1,0) node[anchor=north] {$b$};
          \draw (1,2) node[anchor=south] {$b$};
          
          \draw [->-=0.55] (0,0) -- (0,2);
          \draw [->-=0.55] (0,2) -- (2,2);
          \draw [->-=0.55] (2,0) -- (2,2);
          \draw [->-=0.55] (0,0) -- (2,0);
        \end{tikzpicture}
      };

      \node (d) at (c.east) [anchor=west,xshift=0.5cm]{
        \begin{tikzpicture}[>=stealth]
          \draw (-1,0) node[anchor=east] {$a$};
          \draw (1,0) node[anchor=west] {$b$};
          \fill (0,0) circle (0.066);
          \draw[postaction={decoration={markings,mark=at position 0.75 with {\arrow[scale=2]{>}}},decorate}] (-0.5,0) circle (0.5);
          \draw[postaction={decoration={markings,mark=at position 0.25 with {\arrow[scale=2]{>}}},decorate}] (0.5,0) circle (0.5);
        \end{tikzpicture}
      };
      \path[->,line width=0.5pt] (a) edge (b);
      \path[->,line width=0.5pt] (b) edge (c);
      \path[->,line width=0.5pt] (c) edge (d);
    \end{tikzpicture}
  \end{center}
  To prove this map is indeed a deformation retraction, we use the
  identification of the unit square with the unit disc (in this case the
  boundary of the circle is divided into four arcs with the labeling scheme
  $aba^{-1}b^{-1}$), and using polar coordinates we can let
  $\tilde{F}((r,\theta),t) = (r+t(1-r),\theta)$. Then, let $F = q \circ
  \tilde{F}$ where $q$ is the quotient map shown as the last arrow above.
  $F$ is continuous as $\tilde{F}$ is continuous in each coordinate, and then
  $q$ is continuous. $F$ is also such that $F((r,\theta),0) = (r,\theta)$,
  $F((r,\theta),1) = (1,\theta)$, and $F((1,\theta),t) = (1,\theta)$. Thus, $F$
  is a deformation retraction. The two circles in the last diagram are the
  longitude and meridian circles of the torus by construction, namely since in the
  penultimate diagram we identify the sides marked $a$ to get the meridian circle
  and then the sides marked $b$ to get the longitude circle.
\end{proof}

\setcounter{subsubsection}{2}
\begin{problem}\mbox{}
  \begin{enuma}
    \item Show that the composition of homotopy equivalences $X \to Y$ and $Y \to Z$ is a homotopy equivalence $X \to Z$. Deduce that homotopy equivalence is an equivalence relation.
    \item Show that the relation of homotopy among maps $X \to Y$ is an equivalence relation.
    \item Show that a map homotopic to a homotopy equivalence is homotopy equivalence.
  \end{enuma}
\end{problem}
\begin{remark}
  We remark that homotopy equivalences preserve compositions. For, if $f,g \in \Hom(X,Y)$ and $F(x,t)$ is a homotopy $f \simeq g$, and if $e \colon W \to X$, $h \colon Y \to Z$, then $h(F(e(x),t))$ is a homotopy $e \circ f \circ h \simeq e \circ g \circ h$ since the composition of continuous maps is continuous and since $h(F(e(x),0)) = e \circ f \circ h$ and $h(F(e(x),1)) = e \circ g \circ h$. This allows us to replace maps by homotopy equivalent ones in $(a)$ and $(c)$.
\end{remark}
\begin{proof}[Proof of $(a)$]
  Let $e\colon X \to Y$, $f \colon Y \to X$ such that $e \circ f \simeq \id_X$, $f \circ e \simeq \id_Y$; likewise, let $g \colon Y \to Z$, $h \colon Z \to Y$ such that $g \circ h \simeq \id_Y$, $h \circ g \simeq \id_Z$. Then, since composition of maps is associative,
  \begin{alignat*}{5}
    (e \circ g) \circ (h \circ f) &={}& e \circ (g \circ h) \circ f &\simeq{}& e \circ \id_Y \circ\:f &={}& e \circ f &\simeq{}& \id_X\\
    (h \circ f) \circ (e \circ g) &={}& h \circ (f \circ e) \circ g &\simeq{}& h \circ \id_Y \circ\:g &={}& h \circ g &\simeq{}& \id_Z
  \end{alignat*}
  and so we have a homotopy equivalence $X \simeq Z$ (note we have to use the fact that homotopy is transitive from $(b)$). Since homotopy equivalence is reflexive (take the identity map) and symmetric (change the roles of $X,Z$), homotopy equivalence is an equivalence relation since we have shown transitivity.
\end{proof}
\begin{proof}[Proof of $(b)$]
  Let $f,g,h \in \Hom(X,Y)$. Homotopy is reflexive since we can take $F(x,t) = f(x)$ for all $t \in [0,1]$, and symmetric since if $F(x,t)$ is a homotopy between $f,g$, then $F(x,1-t)$ is a homotopy between $g,f$ by the fact that $1-t$ is continuous and so the composition $F(x,1-t)$ is continuous.
  \par It remains to show homotopy is transitive. Let $F$ be a homotopy between $f,g$ and $G$ be a homotopy between $g,h$. Then, let
  \begin{equation*}
    H(x,t) = \begin{cases}
      F(x,2t) & \text{if}~t \in [0,1/2],\\
      G(x,2t-1) & \text{if}~t \in [1/2,1].\\
    \end{cases}
  \end{equation*}
  This is continuous by the pasting lemma since $H(x,1/2) = F(x,1) = G(x,0) = g(x)$. $H$ is then a homotopy between $f,h$ since $H(x,0) = F(x,0) = f(x)$ while $H(x,1) = G(x,1) = h(x)$. Thus, homotopy is an equivalence relation.
\end{proof}
\begin{proof}[Proof of $(c)$]
  Let $e\colon X \to Y$, $f\colon Y \to X$ such that $e \circ f \simeq \id_X$, $f \circ e \simeq \id_Y$, and let $g\colon X \to Y$ such that $e \simeq g$. Then, $g \circ f \simeq e \circ f \simeq \id_X$, $f \circ g \simeq f \circ e \simeq \id_Y$, and so $g$ is also a homotopy equivalence.
\end{proof}

\setcounter{subsubsection}{8}
\begin{problem}
  Show that a retract of a contractible space is contractible.
\end{problem}
\begin{proof}
  Recall that $X$ is contractible if and only if $\id_X$ is homotopic to a
  constant map mapping to some $x_0 \in X$. Let $F(x,t)$ be this homotopy, and let
  $A$ be our retract, i.e., that there exists a map $r\colon X \to A$ such that
  $r\rvert_A = \id_A$. Then, $(F \circ r)\rvert_A$ is a homotopy between $\id_A$
  and the constant map mapping to $r(x_0)$, for it is a composition of continuous
  maps hence itself continuous, and since $r(F(x,0))\rvert_A = (\id_X \circ
  r)\rvert_A = r\rvert_A = \id_A$ and $r(F(x,1))\rvert_A = r(x_0)$. 
\end{proof}

\setcounter{subsubsection}{13}
\begin{problem}
  Given positive integers $v$, $e$, and $f$ satisfying $v - e + f = 2$, construct a cell structure on $S^2$ having $v$ $0$-cells, $e$ $1$-cells, and $f$ $2$-cells.
\end{problem}
\begin{proof}
  For a given triple $(v,e,f)$, we can construct the following 1-skeleton $X^1$:
  \begin{center}
    \begin{tikzpicture}
      \draw[white] (-3.6,-1) rectangle (6.5,1);
      \foreach \x in {0,...,4}
      \fill (\x,0) circle (.066);
      \path (0,0) edge (2,0);
      \draw [decorate sep={0.25mm}{0.166cm},fill] (2,0) -- (3,0);
      \path (3,0) edge (4,0);
      \foreach \y in {2,3,5,6} {
        \pgfplothandlerlineto
          \pgfplotfunction{\x}{0,1,...,45}{
              \pgfpointxy     
              {cos(\x + 45 * \y - 22.5) * sin(4 * \x)}
              {sin(\x + 45 * \y - 22.5) * sin(4 * \x)}
          }
        \pgfusepath{stroke}
      }
      \draw (-0.6,-0.3) node[anchor=south] {$\vdots$};
      \draw [thick,decoration={brace,mirror,raise=0.3cm},decorate] (0.25,0) -- (3.75,0) node [pos=0.5,anchor=north,yshift=-0.4cm] {$v-1$ segments}; 
      \draw [thick,decoration={brace,raise=1cm},decorate] (0,-1) -- (0,1) node [pos=0.5,anchor=east,xshift=-1.2cm] {$f-1$ petals}; 
    \end{tikzpicture}
  \end{center}
  Each petal is homeomorphic to $S^1$ and each segment is homeomorphic to $I = [0,1]$. This construction works since for each $n$-cell in addition to the 1 0-cell and 1 2-cell that are minimally required to construct $S^2$, each 1-cell that contributes to $e$ must be offset by either a 0-cell that contributes to $v$ or a 2-cell that contributes to $f$.
  \par It now suffices to attach 2-cells such that the resulting space is $S^2$. We see that our 1-skeleton divides the plane up into $f$ regions, one for each interior of a petal and one encompassing the rest of the plane. Attaching a 2-cell to each petal by identifying $\partial e_i^2 \simeq S^1_i$ where $S^1_i$ is one of the petals for $1 \le i \le f-1$, and then attaching a 2-cell such that $\partial e_f^2 \simeq X^1$, i.e., by having the boundary be identified with all of the edges in our 1-skeleton $X^1$, results in $S^2$ as desired.
\end{proof}

\begin{problem}
  Enumerate all the subcomplexes of $S^\infty$, with the cell structure on $S^\infty$ that has $S^n$ as its $n$-skeleton.
\end{problem}
\begin{proof}[Solution]
  Recall that any $S^n$ is constructed as a CW complex consisting of $S^{n-1}$, $e_1^n$, and $e_2^n$, where we identify $\partial e_i^n \simeq S^{n-1}$. Thus, $S^n = \bigcup_{k=0}^n (e_1^k \cup e_2^k)$.
  \par Now suppose $\emptyset \ne A \subseteq S^\infty$ is a subcomplex. Then, $A$ contains some $k$-cell $e_i^k$ for $i=1$ or $2$. Then, since for $A$ to be closed we must have $S^{k-1} = \partial e_i^k \subseteq A$ as well, we see that $A$ must contain every $\ell$-cell for $\ell < k$. Thus, any subcomplex $A$ is in the collection:
  \begin{equation*}
    A \in \{\emptyset,~S^n,~S^\infty\} \cup \{ e_i^n \cup S^{n-1}\mid i=1~\text{or}~2 \},
  \end{equation*}
  for supposing $A \ne \emptyset$, if there is a maximal $n$ such that $e_i^n \subseteq A$ for $i=1$ or $2$, then $A = S^n$ or $e_i^n \cup S^{n-1}$, and if not, then for any $n$, $A$ must contain some $e_i^{n'}$ where $n' > n$, and so $A$ contains all $e_i^n$ by the argument above, i.e., $A = S^\infty$.
\end{proof}

\begin{problem}
  Show that $S^\infty$ is contractible.
\end{problem}
\begin{proof}[Proof following Prop.~$0.16$]
  Let $r_n\colon S^n \times \left[\frac{1}{2^{n+1}},\frac{1}{2^n}\right] \to
  S^n$ be the homotopy pushing the equator $S^{n-1}$ to a point $p \in e_2^n$; note
  that this also contracts $e_2^n$ to the point $p$. Then, combining the
  homotopies $r_n$ gives a homotopy $r\colon S^\infty \times I \to S^\infty$
  between the identity map and a constant map. There is no continuity issues at
  $t=0$ since $r$ is continuous on $S^n \times I$, since it is stationary on
  $\left[ 0,\frac{1}{2^{n+1}} \right]$, and since CW complexes have the weak
  topology with respect to skeleta.
\end{proof}
\begin{proof}[Proof following \emph{Ex.~1B.3}]
  Let $I = [0,1]$. Define
  \begin{align*}
    f \colon \RR^\infty \times I &\to \RR^\infty\\
    (x_1,x_2,\ldots) \times t &\mapsto (1-t)(x_1,x_2,\ldots) + t(0,x_1,x_2,\ldots)
  \end{align*}
  Note $f$ is continuous by \cite[Thm.~19.6]{Mun00} since it is continuous in
  each coordinate. $f_t$ takes nonzero vectors to nonzero vectors for all
  $t \in I$, hence $f_t/\lvert f_t \rvert$ gives a homotopy from the identity map
  of $S^\infty$ to the map $(x_1,x_2,\ldots) \mapsto (0,x_1,x_2,\ldots)$.
  Next, define
  \begin{align*}
    g \colon \RR^\infty \times I &\to \RR^\infty\\
    (x_1,x_2,\ldots) \times t &\mapsto (1-t)(0,x_1,x_2,\ldots) + t(1,0,0,\ldots)
  \end{align*}
  $g$ is continuous for the same reason as for $f$, and $g_t$ takes nonzero
  vectors to nonzero vectors for all $t\in I$. Thus, $g_t/\lvert g_t \rvert$
  gives a homotopy from the map $(x_1,x_2,\ldots) \mapsto (0,x_1,x_2,\ldots)$ to
  a constant map. Composing these two homotopies together gives a homotopy
  between the identity map and a constant map, so $S^\infty$ is contractible.
\end{proof}

\begin{problem}
  Construct a $2$-dimensional CW complex that contains both an annulus $S^1 \times I$ and a M\"obius band as deformation retracts.
\end{problem}
\begin{proof}[Solution]
  Consider the CW complex and the maps drawn with small arrows below:
  \begin{center}
    \begin{tikzpicture}[line cap=round,line join=round,x=1.0cm,y=0.9cm]
      \node (a) at (0,0) {
        \begin{tikzpicture}[>=stealth]
          \draw[white] (0,0) -- (2.5,0);
          \fill [gray] (0,0) -- (0,2) -- (1.8,2.6) -- (1.8,0.6);
          \fill [gray] (0.9,0.3) -- (0.9,2.3) -- (2.3,1.8) -- (2.3,-0.2);
          \draw [->-=0.55] (0,2) -- (0.9,2.3);
          \draw [->-=0.55] (0.9,2.3) -- (1.8,2.6);
          \draw [->-=0.55] (0.9,0.3) -- (0,0);
          \draw [->-=0.55,dashed] (1.8,0.6) -- (0.9,0.3);
          \draw [dashed] (1.8,0.6) -- (1.8,2);
          \draw (1.8,1.98) -- (1.8,2.6);
          \draw (0,0) -- (0,2);
          \draw [->-=0.55] (0.9,0.3) -- (2.3,-0.2);
          \draw [->-=0.55] (0.9,2.3) -- (2.3,1.8);
          \draw (0.9,0.3) -- (0.9,2.3);
          \draw (2.3,1.8) -- (2.3,-0.2);

          \draw (0.45,0.15) node[anchor=south] {$b$};
          \draw (0.45,2.15) node[anchor=south] {$a$};
          \draw (1.35,0.45) node[anchor=south] {$a$};
          \draw (1.35,2.45) node[anchor=south] {$b$};

          \draw (1.6,2.05) node[anchor=north] {$c$};
          \draw (1.6,0.05) node[anchor=north] {$c$};
        \end{tikzpicture}
      };
      \node (b) at (a.north east) [anchor=west,xshift=0.5cm] {
        \begin{tikzpicture}[>=stealth]
          \draw[white] (-0.2,0) -- (2.5,0);
          \fill [gray] (0,0) -- (0,2) -- (1.8,2.6) -- (1.8,0.6);
          \fill [gray] (0.9,0.3) -- (0.9,2.3) -- (2.3,1.8) -- (2.3,-0.2);
          \draw [->-=0.55] (0,2) -- (0.9,2.3);
          \draw [->-=0.55] (0.9,2.3) -- (1.8,2.6);
          \draw [->-=0.55] (0.9,0.3) -- (0,0);
          \draw [->-=0.55,dashed] (1.8,0.6) -- (0.9,0.3);
          \draw [dashed] (1.8,0.6) -- (1.8,2);
          \draw (1.8,1.98) -- (1.8,2.6);
          \draw (0,0) -- (0,2);
          \draw [->-=0.55] (0.9,0.3) -- (2.3,-0.2);
          \draw [->-=0.55] (0.9,2.3) -- (2.3,1.8);
          \draw [->--=0.55] (2.3,1.4) -- (0.9,1.9);
          \draw [->--=0.55] (2.3,1.0) -- (0.9,1.5);
          \draw [->--=0.55] (2.3,0.6) -- (0.9,1.1);
          \draw [->--=0.55] (2.3,0.2) -- (0.9,0.7);
          \draw (0.9,0.3) -- (0.9,2.3);
          \draw (2.3,1.8) -- (2.3,-0.2);
        \end{tikzpicture}
      };
      \node (c) at (b.east) [anchor=west,xshift=0.5cm] {
        \begin{tikzpicture}[>=stealth]
          \draw[white] (-0.2,0) -- (2.3,0);
          \fill [gray] (0,0) -- (0,2) -- (1.8,2.6) -- (1.8,0.6);
          \draw [->-=0.55] (0,2) -- (0.9,2.3);
          \draw [->-=0.55] (0.9,2.3) -- (1.8,2.6);
          \draw [->-=0.55] (0.9,0.3) -- (0,0);
          \draw [->-=0.55] (1.8,0.6) -- (0.9,0.3);
          \draw (1.8,0.6) -- (1.8,2.6);
          \draw (0,0) -- (0,2);
          \draw (0.9,0.3) -- (0.9,2.3);
          
          \draw (0.45,0.15) node[anchor=north] {$b$};
          \draw (0.45,2.15) node[anchor=south] {$a$};
          \draw (1.35,0.45) node[anchor=north] {$a$};
          \draw (1.35,2.45) node[anchor=south] {$b$};
        \end{tikzpicture}
      };
      \path[->,line width=0.5pt] (a) edge (b);
      \path[->,line width=0.5pt] (b) edge (c);

      \node (d) at (a.south east) [anchor=west,xshift=0.5cm] {
        \begin{tikzpicture}[>=stealth]
          \draw[white] (-0.2,0) -- (2.5,0);
          \fill [gray] (0,0) -- (0,2) -- (1.8,2.6) -- (1.8,0.6);
          \fill [gray] (0.9,0.3) -- (0.9,2.3) -- (2.3,1.8) -- (2.3,-0.2);
          \draw [->-=0.55] (0,2) -- (0.9,2.3);
          \draw [->-=0.55] (0.9,2.3) -- (1.8,2.6);
          \draw [->-=0.55] (0.9,0.3) -- (0,0);
          \draw [->-=0.55,dashed] (1.8,0.6) -- (0.9,0.3);

          \draw (1.8,2.2) -- (1.5,2.1);
          \draw [->--=0.55,dashed] (1.8,2.2) -- (0.9,1.9);
          \draw [->--=0.55,dashed] (1.8,1.8) -- (0.9,1.5);
          \draw [->--=0.55,dashed] (1.8,1.4) -- (0.9,1.1);
          \draw [->--=0.55,dashed] (1.8,1.0) -- (0.9,0.7);

          \draw [->--=0.55] (0,1.6) -- (0.9,1.9);
          \draw [->--=0.55] (0,1.2) -- (0.9,1.5);
          \draw [->--=0.55] (0,0.8) -- (0.9,1.1);
          \draw [->--=0.55] (0,0.4) -- (0.9,0.7);

          \draw [dashed] (1.8,0.6) -- (1.8,2);
          \draw (1.8,1.98) -- (1.8,2.6);
          \draw (0,0) -- (0,2);
          \draw [->-=0.55] (0.9,0.3) -- (2.3,-0.2);
          \draw [->-=0.55] (0.9,2.3) -- (2.3,1.8);
          \draw (0.9,0.3) -- (0.9,2.3);
          \draw (2.3,1.8) -- (2.3,-0.2);
        \end{tikzpicture}

      };
      \node (e) at (d.east) [anchor=west,xshift=0.5cm] {
        \begin{tikzpicture}[>=stealth]
          \draw[white] (0.45,0) -- (2.3,0);
          \fill [gray] (0.9,0.3) -- (0.9,2.3) -- (2.3,1.8) -- (2.3,-0.2);
          \draw [->-=0.55] (0.9,0.3) -- (2.3,-0.2);
          \draw [->-=0.55] (0.9,2.3) -- (2.3,1.8);
          \draw (0.9,0.3) -- (0.9,2.3);
          \draw (2.3,1.8) -- (2.3,-0.2);

          \draw (1.6,2.05) node[anchor=south] {$c$};
          \draw (1.6,0.05) node[anchor=north] {$c$};
        \end{tikzpicture}
      };
      \path[->,line width=0.5pt] (a) edge (d);
      \path[->,line width=0.5pt] (d) edge (e);
    \end{tikzpicture}
  \end{center}
  \par In the top row, since the segment labeled $c$ is homeomorphic to $[0,1]$, we see that if the unit square containing $c$ has the coordinate along $c$ as the $x$ coordinate, and the other edge as the $y$ coordinate, our map is given by $F((x,y),t) = ((1-t)x,y)$ in this unit square, and the identity on the unit square containing $a,b$. $F$ is well-defined since it acts the same on the identified edges. $F$ is continuous in each coordinate hence continuous, and is such that $F((x,y),0) = (x,y)$, $F((x,y),1) = (0,y)$, and $F((0,y),t) = (0,y)$; thus, $F$ on the entire complex is continuous by the pasting lemma. $F$ is therefore a deformation retract to the M\"obius strip defined by the unit square containing $a,b$.
  \par In the bottom row, let the horizontal edges be the $x$ coordinate axis with $x \in [0,1]$, and the vertical edges as the $y$ coordinate. Then, our map is given by $F((x,y),t) = (x + t(1/2-x),y)$ in this unit square, and the identity on the unit square containing $c$. $F$ is well-defined since it acts the same on the identified edges. $F$ is continuous in each coordinate hence continuous, and is such that $F((x,y),0) = (x,y)$, $F((x,y),1) = (0,y)$, and $F((1/2,y),t) = (1/2,y)$; thus, $F$ on the entire complex is continuous by the pasting lemma. $F$ is therefore a deformation retract to the annulus defined by the unit square containing $c$.
\end{proof}

\begin{problem}
  Show that $S^1 \ast S^1 = S^3$, and more generally $S^m \ast S^n = S^{m+n+1}$.
\end{problem}
\begin{proof}
  We first want to show $S^n = \bigast_{i=0}^n S^0$. This trivially holds for $n=0$, and so we consider the inductive case. Recall $S^{n+1} = S(S^n) = S\left( \bigast_{i=0}^n S^0 \right)$, and so $X \ast S^0 = S(X)$ implies $S^{n+1} = \bigast_{i=0}^{n+1} S^0$. Thus,
  \begin{equation*}
    S^m \ast S^n = \left( \bigast_{i=0}^{m} S^0 \right) \ast \left( \bigast_{k=0}^{n} S^0 \right) = \bigast_{i=0}^{m+n+1} S^0 = S^{m+n+1}
  \end{equation*}
  since $\ast$ is associative as remarked on p.~9. In particular, $S^1 \ast S^1 = S^3$.
\end{proof}

\setcounter{subsubsection}{19}
\begin{problem}
  Show that the subspace $X \subset \mathbb{R}^3$ formed by a Klein bottle intersecting itself in a circle, as shown in the figure, is homotopy equivalent to $S^1 \vee S^1 \vee S^2$.
\end{problem}
\begin{proof}
  Using the characterization of the Klein bottle as a quotient of a square, we have that $X$ is the quotient of the following square:
  \begin{center}
    \begin{tikzpicture}[line cap=round,line join=round,>=stealth,x=1.0cm,y=1.0cm]
      \fill [gray] (0,0) rectangle (2,2);
      \draw (0,1) node[anchor=east] {$a$};
      \draw (2,1) node[anchor=west] {$a$};
      \draw (1,0) node[anchor=north] {$b$};
      \draw (1,2) node[anchor=south] {$b$};
      \draw [->-=0.55] (0,0) -- (0,2);
      \draw [->-=0.55] (2,2) -- (0,2);
      \draw [->-=0.55] (2,0) -- (2,2);
      \draw [->-=0.55] (0,0) -- (2,0);
      \draw (1,1) node[anchor=north] {$b$};
      \draw [decoration={markings, mark=at position 0.77 with {\arrow[scale=2]{>}}}, postaction={decorate}] (1,1) circle(0.5);
    \end{tikzpicture}
  \end{center}
  We first deformation retract the disc in the center to a point:
  \begin{center}
    \begin{tikzpicture}[line cap=round,line join=round,>=stealth,x=1.0cm,y=1.0cm]
      \fill [gray] (0,0) rectangle (2,2);
      \draw (0,1) node[anchor=east] {$a$};
      \draw (2,1) node[anchor=west] {$a$};
      \draw (1,0) node[anchor=north] {$b$};
      \draw (1,2) node[anchor=south] {$b$};
      \draw [->-=0.55] (0,0) -- (0,2);
      \draw [->-=0.55] (2,2) -- (0,2);
      \draw [->-=0.55] (2,0) -- (2,2);
      \draw [->-=0.55] (0,0) -- (2,0);
      \draw (1,1) node[anchor=north] {$b$};
      \fill (1,1) circle (0.08);
    \end{tikzpicture}
  \end{center}
  But then, since the two horizontal edges are identified with this center point, we can contract them to points in our diagram to get the space
  \begin{center}
    \begin{tikzpicture}[line cap=round,line join=round,>=stealth,x=1.0cm,y=1.0cm]
      \fill [gray] (0,0) circle (1);
      \draw[decoration={markings, mark=at position 0.52 with {\arrow[scale=2]{<}}}, postaction={decorate}] (0,0) circle (1);
      \draw[decoration={markings, mark=at position 0.01 with {\arrow[scale=2]{>}}}, postaction={decorate}] (0,0) circle (1);
      \draw (-1,0) node[anchor=east] {$a$};
      \draw (1,0) node[anchor=west] {$a$};
      \fill (0,1) circle (0.08);
      \fill (0,0) circle (0.08);
      \fill (0,-1) circle (0.08);
      %\draw (0,1) node[anchor=east] {$a$};
      %\draw (2,1) node[anchor=west] {$a$};
      %\draw (1,0) node[anchor=north] {$b$};
      %\draw (1,2) node[anchor=south] {$b$};
      %\draw [->-=0.55] (0,0) -- (0,2);
      %\draw [->-=0.55] (2,2) -- (0,2);
      %\draw [->-=0.55] (2,0) -- (2,2);
      %\draw [->-=0.55] (0,0) -- (2,0);
      %\draw (1,1) node[anchor=north] {$b$};
      %\draw [decoration={markings, mark=at position 0.77 with {\arrow[scale=2]{>}}}, postaction={decorate}] (1,1) circle(0.5);
    \end{tikzpicture}
  \end{center}
  where the three labeled points are identified. This is the same as a sphere with three points identified, and so, attaching $1$-cells as in Example $0.8$, we see that $X$ is homotopy equivalent to $S^1 \vee S^1 \vee S^2$.
\end{proof}

%\begin{problem}
%  If $X$ is a connected Hausdorff space that is a union of a finite number of $2$-spheres, any two of which intersect in at most one point, show that $X$ is homotopy equivalent to a wedge sum of $S^1$'s and $S^2$'s.
%\end{problem}
%\begin{proof}
%  First, every sphere must intersect with at least one other sphere, for suppose
%  not, and $S^2_i$ does not intersect with any other sphere. 
%  
%  Then, $X$ would be disconnected, for $S^1_i$ is both open and closed in $S^1_i$ and hence open and closed in $X$; thus, $S^1_i \amalg X \setminus S^1_i$ is a separation of $X$.
%  \par We proceed by strong induction on number of spheres. The claim is trivial for one sphere, and so we consider the case for $n+1$ spheres. Suppose $S^2_{n+1}$ has intersection points $\{p_j\}_{1 \le j \le J}$ with unions of spheres $X_j$. Then, identifying the $p_j$'s using $1$-cells as in Example 0.8, we see that $S^2_{n+1} \cup \bigcup_{j=1}^J X_j$ is homotopy equivalent to $S^2_{n+1} \wedge \bigcup_{j=1}^J X_j \wedge \bigwedge_{j=1}^J S_1$. By inductive hypothesis, each $X_j$ is a wedge of circles and spheres, and so we are done.
%\end{proof}

\setcounter{subsubsection}{23}
\begin{problem}
  Let $X$ and $Y$ be CW complexes with $0$-cells $x_0$ and $y_0$. Show that the quotient spaces $X \ast Y/(X \ast \{y_0\} \cup \{x_0\} \ast Y)$ and $S(X \vee Y)/S(\{x_0\} \wedge \{y_0\})$ are homeomorphic, and deduce that $X \ast Y \simeq S(X \wedge Y)$.
\end{problem}
\begin{proof}
  Recall that
  \begin{equation*}
    X \ast Y = \frac{X \times Y \times I}{(x,y_1,0) \sim (x,y_2,0), (x_1,y,1)\sim(x_2,y,1)}.
  \end{equation*}
  Under this identification, $X \times \{y_0\} \times I \cup \{x_0\} \times Y \times I$ maps to $X \ast \{y_0\} \cup \{x_0\} \ast Y$, and so we have
  \begin{equation*}
    \frac{X \ast Y}{X \ast \{y_0\} \cup \{x_0\} \ast Y} \cong \frac{X \times Y \times I}{(X \times \{y_0\} \cup \{x_0\} \times Y) \times I}.
  \end{equation*}
  \par Note we also have
  \begin{align*}
    \frac{X \times Y \times I}{(X \times \{y_0\} \cup \{x_0\} \times Y) \times I} &\cong \frac{X \times Y \times I}{(X \vee Y) \times I}
    \cong \frac{X \times Y}{X \vee Y} \times I \cong (X \wedge Y) \times I\\
    &\cong \frac{X \wedge Y}{\{x_0\} \wedge \{y_0\}} \times I \cong \frac{(X \wedge Y) \times I}{(\{x_0\} \wedge \{y_0\}) \times I}.
  \end{align*}
  Recall that
  \begin{equation*}
    S(X) = \frac{X \times I}{(x,0) \sim (x',0), (x,1) \sim (x',1)}.
  \end{equation*}
  Under this identification for $X \wedge Y$, we see that $(X \wedge Y) \times \{0\}$ and $(X \wedge Y) \times \{1\}$ collapse to two points. Thus, $(\{x_0\} \wedge \{y_0\}) \times I$ maps to $S(\{x_0\} \wedge \{y_0\})$. By transitivity of homeomorphisms, we see
  \begin{equation*}
    \frac{X \ast Y}{X \ast \{y_0\} \cup \{x_0\} \ast Y} \cong \frac{S(X \wedge Y)}{S(\{x_0\} \wedge \{y_0\})}.
  \end{equation*}
  \par Finally, both $(X \ast Y,X \ast \{y_0\} \cup \{x_0\} \ast Y)$ and $(S(X \vee Y),S(\{x_0\} \wedge \{y_0\}))$ are CW pairs, and so they satisfy the homotopy extension property by Proposition 0.16, and therefore are homotopy equivalent to their quotient spaces by the subcomplex by Proposition 0.17. This gives that
  \begin{equation*}
    X \ast Y \simeq \frac{X \ast Y}{X \ast \{y_0\} \cup \{x_0\} \ast Y} \cong \frac{S(X \wedge Y)}{S(\{x_0\} \wedge \{y_0\})} \simeq S(X \wedge Y),
  \end{equation*}
  and so $X \ast Y \simeq S(X \wedge Y)$.
\end{proof}
\endgroup

\section{The Fundamental Group}
\subsection{Basic Constructions}
\setcounter{subsubsection}{7}
\begin{problem}
  Does the Borsuk-Ulam theorem hold for the torus? In other words, for every map $f\colon S^1 \times S^1 \to \mathbb{R}^2$ must there exist $(x,y) \in S^1 \times S^1$ such that $f(x,y) = f(-x,-y)$?
\end{problem}
\begin{proof}[Solution]
  We consider the torus $S^1 \times S^1$ embedded into $\mathbb{R}^3$ via an
  immersion $\iota\colon S^1 \times S^1 \hookrightarrow \mathbb{R}^3$, such that
  the torus is symmetric about the $z$-axis and across the $xy$-plane. Then, let
  $f(x,y,z) = (x,y)$ be the projection map from torus to the $xy$-plane. It is
  continuous since the projection $\mathbb{R}^3 \to \mathbb{R}^2$ is continuous,
  and since the topology on the torus is then the same as the subspace topology
  inherited from $\mathbb{R}^3$. Since $(f \circ \iota)(-x,-y) \ne
  (f \circ \iota)(x,y)$, we see that the Borsuk-Ulam theorem does not hold for this
  map.
\end{proof}

\begin{problem}
  Let $A_1$, $A_2$, $A_3$ be compact sets in $\mathbb{R}^3$. Use the Borsuk-Ulam theorem to show that there is one plane $P \subset \mathbb{R}^3$ that simultaneously divides each $A_i$ into two planes of equal measure.
\end{problem}
\begin{proof}
  Since $A_3$ is compact, it is closed and bounded; thus, we can assume without loss of generality that $A_3$ lies in the region $z > 0$ by changing coordinates. Then, embed $S^2$ into $\mathbb{R}^3$ such that the center of $S^2$ and the origin in $\mathbb{R}^3$ coincide; every unit vector $s \in S^2$ then defines a direction in $\mathbb{R}^3$. We first note that every pair $(s,r) \in S^2 \times \mathbb{R}$ defines a unique plane $P(s,r)$ with normal vector $s$ distance $r$ from the origin. Note that negative $r$ denotes that the plane lies in the negative direction with respect to the direction defined by $s$. Thus, we can define $\rho(s,r) = \mu(A_3^+(s,r))$, the measure of the subset of $A_3$ on the positive side of the plane $P(s,r)$. $\rho(s,r)$ is continuous in $r$ since $\mu$ is continuous from above and below.
  \par Now fix $s$. Since $\rho(s,r) = \mu(A_3)$ for $r$ sufficiently small and $\rho(s,r) = 0$ for $r$ sufficiently large, by the intermediate value theorem there exists for this fixed $s$ some $r_0$ such that $\rho(s,r_0) = \frac{1}{2} \mu(A_3)$. If there are multiple such $r_0$ that form an interval $I$, then we take $r_0$ to be the midpoint of the interval $I$. Thus, for each $s$ we can define a unique plane $P(s)$ such that $\mu(A_3^+(s,r)) = \frac{1}{2} \mu(A_3)$.
  \par We now let $f(s) = (\mu(A_1^+(s)),\mu(A_2^+(s)))$, where $A_i^+(s)$ denotes the subset of $A_i$ on the positive side of $P(s)$, is continuous. $f$ is continuous since small perturbation in $s$ will only lead to small changes in our plane $P(s)$, and therefore in $\mu(A_1^+(s))$ and $\mu(A_2^+(s))$. Note that since $P(s) = -P(-s)$, i.e., the plane defined for $-s$ is the same as that defined for $s$, except oriented in the other direction, we have the equality $f(s) + f(-s) = (\mu(A_1),\mu(A_2))$.
  \par By the Borsuk-Ulam theorem, there now exists $s_0 \in S^2$ such that $f(s_0) = f(-s_0)$. But by the equality $f(s) + f(-s) = (\mu(A_1),\mu(A_2))$, we see that then $f(s_0) = f(-s_0) = \frac{1}{2} (\mu(A_1),\mu(A_2))$, and so our plane $P(s_0)$ divides the sets $A_1,A_2,A_3$ into halves of equal measure.
\end{proof}

\setcounter{subsubsection}{15}
\begin{problem}
  Show that there are no retractions $r\colon X \to A$ in the following cases:
  \begin{enuma}
    \item $X = \mathbb{R}^3$ with $A$ any subspace homeomorphic to $S^1$.
    \item $X = S^1 \times D^2$ with $A$ its boundary torus $S^1 \times S^1$.
    \item $X = S^1 \times D^2$ and $A$ the circle shown in the figure.
    \item $X = D^2 \vee D^2$ with $A$ its boundary $S^1 \vee S^1$.
    \item $X$ a disk with two points on its boundary identified and $A$ its boundary $S^1 \vee S^1$.
    \item $X$ the M\"obius band and $A$ its boundary circle.
  \end{enuma}
\end{problem}
%\begin{remark}
%  We will repeatedly use Proposition $1.17$, i.e., if $X$ retracts onto $A$, then $i_* \colon \pi_1(A,x_0) \to \pi_1(X,x_0)$ induced by $i\colon A \hookrightarrow X$ is injective.
%\end{remark}
\begin{proof}[Proof of $(a)$]
  Since $\mathbb{R}^3$ is convex, $\pi_1(X) = 0$ by Example 1.4. Also, $\pi_1(A) = \mathbb{Z}$ by Theorem $1.7$. But $\mathbb{Z}$ does not inject into $0$, contradicting Proposition $1.17$.
\end{proof}
\begin{proof}[Proof of $(b)$]
  Since $X$ deformation retracts to its center circle, $\pi_1(X) = \mathbb{Z}$ by Proposition $1.17$. However, $\pi_1(A) = \mathbb{Z}^2$ by Example $1.13$. But $\mathbb{Z}^2$ does not inject into $\mathbb{Z}$, for every non-trivial subgroup of $\mathbb{Z}$ is isomorphic to $\mathbb{Z}$, and free $\mathbb{Z}$-modules of different rank are non-isomorphic, contradicting Proposition $1.17$.
\end{proof}
\begin{proof}[Proof of $(c)$]
  Recall that $X$ deformation retracts to its center circle $C$, and so we have the chain $A \overset{i}{\hookrightarrow} X \overset{p}{\to} C$ which induces the maps $\pi_1(A) \overset{i_*}{\to} \pi_1(X) \overset{p_*}{\to} \pi_1(C)$, where $p_*$ is an isomorphism by Proposition $1.17$. But if $f$ is a representative of the generator of $\pi_1(A) = \mathbb{Z}$, we see that $(i \circ p)(f)$ is nulhomotopic since it has the same start and end points when lifted up into $\mathbb{R}$ as in Theorem $1.7$, and so $(i_* \circ p_*)(1) = 0$ the constant loop. Hence, $i_*$ is not an injection, contradicting Proposition $1.17$.
\end{proof}
\begin{proof}[Proof of $(d)$]
  By Example $1.21$, $\pi_1(X) = \pi_1(D^2) \ast \pi_1(D^2) = 0$ by Example $1.4$ since $D^2$ is convex. However, $\pi_1(A) = \mathbb{Z} \ast \mathbb{Z}$ again by Example $1.21$. $\mathbb{Z} \ast \mathbb{Z}$ does not inject into $0$, contradicting Proposition $1.17$.
\end{proof}
\begin{proof}[Proof of $(e)$]
  We see that $D^2$ with a $1$-cell attached to two points on the boundary is homotopy equivalent to $X$, since the $1$-cell is contractible. On the other hand, since $D^2$ is contractible, $X$ is homotopy equivalent to $S^1$. Thus, $\pi_1(X) = \mathbb{Z}$, while $\pi_1(A) = \mathbb{Z} \ast \mathbb{Z}$ by Example $1.21$. Since $\mathbb{Z} \ast \mathbb{Z}$ is not abelian while all subgroups of $\mathbb{Z}$ are abelian, $\mathbb{Z} \ast \mathbb{Z}$ does not inject into $\mathbb{Z}$, contradicting Proposition $1.17$.
\end{proof}
\begin{proof}[Proof of $(f)$]
  Recall that $X$ deformation retracts to its center circle $C$, and so we have the chain $A \overset{i}{\hookrightarrow} X \overset{p}{\to} C$ which induces the maps $\pi_1(A) \overset{i_*}{\to} \pi_1(X) \overset{p_*}{\to} \pi_1(C)$, where $p_*$ is an isomorphism by Proposition $1.17$. But if $f$ is a representative of a generator of $\pi_1(A) = \mathbb{Z}$, we see that $(i \circ p)(f)$ maps to $g^2$ where $g$ is a representative of the generator of $\pi_1(C) = \mathbb{Z}$, and so $(i_* \circ p_*)(1) = 2$, and in particular $i_*(1) = 2$. This contradicts that $i_* \circ r_* = \id_{\pi(A)}$ since every homomorphism $\mathbb{Z} \to \mathbb{Z}$ is of the form $n \mapsto kn$ for some $k \in \mathbb{Z}$, and so there is no homomorphism such that $2 \mapsto 1$.
\end{proof}

\subsection{Van Kampen's Theorem}
\setcounter{subsubsection}{3}
\begin{problem}
  Let $X \subset \mathbb{R}^3$ be the union of $n$ lines through the origin. Compute $\pi_1(\mathbb{R}^3 - X)$.
\end{problem}
\begin{proof}
  We will perform a series of deformation retractions. For any point $p \in \mathbb{R}^3 - X$, we can retract it onto $S^2$ by moving it along the line joining $0$ and $p$. This is clearly continuous and is moreover a retraction since points already on $S^2$ do not move. The retract of $\mathbb{R}^3 - X$ is then the $2n$-punctured sphere since each line that passes through the origin punctures the sphere in two locations; let $\{x_i\}_{1 \le i \le 2n}$ be the set of missing points. Then, picking $x_{2n}$ as our pole, we can stereographically project $S^2 - \{x_i\}_{1 \le i \le 2n}$ onto $\mathbb{R}^2 - \{x_i'\}_{1 \le i \le 2n-1}$, which as we recall is a homeomorphism. Finally, we can deformation retract the $(2n-1)$-punctured plane to $\bigvee_{i=1}^{2n-1} S^1$ in the same way we usually retract the doubly punctured plane to $S^1 \vee S^1$, and so $\pi_1(\mathbb{R}^3 - X) = \bigast_{i=1}^{2n-1} \mathbb{Z}$ by Example $1.21$.
\end{proof}

\setcounter{subsubsection}{7}
\begin{problem}
  Compute the fundamental group of the space obtained from two tori $S^1 \times S^1$ by identifying a circle $S^1 \times \{x_0\}$ in on one torus with the corresponding circle $S^1 \times \{x_0\}$ in the other torus.
\end{problem}
\begin{proof}
  Let $X$ be this topological space. Recall that the torus has $1$-skeleton $S^1
  \vee S^1$; by identifying one circle with that of another torus, we get the
  $1$-skeleton $X^1$ of $X$ to be $S^1 \vee S^1 \vee S^1$. Thus $\pi_1(X^1) =
  \mathbb{Z} \ast \mathbb{Z} \ast \mathbb{Z}$; let $a,b,c$ be representatives of
  the generators of each copy of $\mathbb{Z}$. If $a$ is the generator of the
  circle that is identified from the two tori, then the normal subgroup $N$ on
  p.~50 is the normalizer of the subgroup generated by all elements of the form
  $aba^{-1}b^{-1}$ and $aca^{-1}c^{-1}$, for each $2$-cell is attached along the
  loop given by these two generators. By Proposition $1.26$, $\pi_1(X) \approx
  \pi_1(X_1)/N = \mathbb{Z} \times (\mathbb{Z} \ast \mathbb{Z})$.
\end{proof}

\setcounter{subsubsection}{8}
\begin{problem}
  In the surface $M_g$ of genus $g$, let $C$ be the circle that separates $M_g$ into two compact subsurfaces $M_h'$ and $M_k'$ obtained from the closed surfaces $M_h$ and $M_k$ by deleting an open disk from each. Show that $M_h'$ does not retract onto its boundary circle $C$, and hence $M_g$ does not retract onto $C$. But show that $M_g$ does retract onto the nonseparating circle $C'$ in the figure.
\end{problem}
\begin{proof}
  Suppose $C$ is a retract. We have the following commutative diagram:
  \begin{center}
    \begin{tikzcd}[row sep=tiny]
      {}& \pi_1(M_h')\arrow[twoheadrightarrow]{dd}{q}\\
      \pi_1(C)\arrow[hookrightarrow]{ur}{\iota_*}\arrow{dr}[swap]{\tilde{\iota}_*}\\
      & H_1(M_h')
    \end{tikzcd}
  \end{center}
  Proposition $1.17$ implies $\iota_*$ is injective. We claim $\tilde{\iota}_*$
  is also. Since $\iota_*$ is injective, $\tilde{\iota}_*(x) = 0$ for $x \in
  \pi_1(C)$ if and only if $x$ is in the derived subgroup of $\pi_1(M_h')$. But
  this implies that $x=0$, for any element in the derived subgroup of
  $\pi_1(M_h')$ is of the form $aba^{-1}b^{-1}$, and the only element in the
  image of $\iota_*$ of this form is $\iota_*(0) = 1$, since $\pi_1(C) =
  \mathbb{Z}$. Thus, $\tilde{\iota}_* = \iota_* \circ q$ is injective.
  \par We claim this is a contradiction. We first draw the cell structure of $M_h'$:
  \begin{center}
    \begin{tikzpicture}[line cap=round,line join=round,>=triangle 45,x=1.0cm,y=1.0cm]
      \fill[fill=gray] (0,0) -- (0,1) -- (0.71,1.71) -- (1.71,1.71) -- (2.41,1) -- (2.41,0) -- (1.71,-0.71) -- (0.71,-0.71) -- (0,0);
      \path
        (0,0) edge[<-] node[anchor=east] {$b_g^{-1}$} (0,1)
        (0,1) edge[->] node[anchor=south east] {$a_1$} (0.71,1.71)
        (0.71,1.71) edge[->] node[anchor=south] {$b_1$} (1.71,1.71)
        (1.71,1.71) edge[<-] node[anchor=south west] {$a_1^{-1}$} (2.41,1)
        (2.41,1) edge[<-] node[anchor=west] {$b_1^{-1}$} (2.41,0)
        (2.41,0) edge[dashed] (1.71,-0.71)
        (0.71,-0.71) edge[dashed] (1.71,-0.71)
        (0.71,-0.71) edge[<-] node[anchor=north east] {$a_g^{-1}$} (0,0);
      \pgfplothandlerlineto
        \pgfplotfunction{\x}{0,1,...,45}{
            \pgfpointxy     
            {1.25*cos(\x - 45 * 0.5 - 22.5) * sin(4 * \x)}
            {1.25*sin(\x - 45 * 0.5 - 22.5) * sin(4 * \x) + 1}
        }
      \pgfsetfillcolor{white}
      \pgfusepath{fill,stroke}
      \pgfsetfillcolor{black}
      \path (0.82,0.84) edge[->] (0.72,0.89);
      \draw (0.9,0.9) node[anchor=west] {$c$};
    \end{tikzpicture}
  \end{center}
  Note that $c$ is the generator of $\pi_1(C)$. We see that the 2-cell is attached using the scheme $c[a_1,b_1]\cdots[a_g,b_g]$, and so the fundamental group is given by
  \begin{equation*}
    \pi_1(M_h') = \Braket{a_1,\ldots,a_g,b_1,\ldots,b_g,c | c\prod_{i=1}^g [a_i,b_i]}
  \end{equation*}
  which has the abelianization
  \begin{equation*}
    H_1(M_h') = \braket{a_1,\ldots,a_g,b_1,\ldots,b_g,c | c} = \braket{a_1,\ldots,a_g,b_1,\ldots,b_g},
  \end{equation*}
  since $\prod_{i=1}^g [a_i,b_i] \mapsto 0$ under the abelianization map. This implies that $\tilde{\iota}_*(c) = 0$, contradicting injectivity. Thus, $M_g$ does not retract onto $C$, for the restriction of this retraction to $M_h'$ would then be a retract $M_h' \longrightarrow C$.
  \par We now claim that $M_g$ retracts onto the nonseparating circle $C'$. We consider part of our cell structure in the following diagram:
  \begin{center}
    \begin{tikzpicture}[line cap=round,line join=round,>=triangle 45,x=1cm,y=1cm]
      \fill[fill=gray] (-0.4,0) -- (1,0) -- (2,1) -- (3.4,1) -- (3.4,2.4) -- (2,2.4) -- (1,3.4) -- (-0.4,3.4) -- (-0.4,0);
      \path
        (-0.4,0) edge[->] (1,0)
        (1,0) edge[->] node[anchor=north west] {$b_i$} (2,1)
        (2,1) edge[->] node[anchor=north] {$a_i$} (3.4,1)
        (3.4,1) edge[<-] node[anchor=west] {$b_i$} (3.4,2.4)
        (3.4,2.4) edge[<-] node[anchor=south] {$a_i$} (2,2.4)
        (2,2.4) edge[->] (1,3.4)
        (1,3.4) edge[->] (-0.4,3.4)
        (2,2.4) edge[dashed] (2,1);
    \end{tikzpicture}
  \end{center}
  where $a_i$ is identified with the circle $C'$. We claim that we can extend a retraction projecting the top copy of $a_i$ to the bottom copy in the square on the right side can be extended to a retraction of the whole complex. This retracts the left dashed edge of the square into one point $p$, and the same applies to $b_i$ on the other side of this square. We can map the rest of the polygon outside of the right square onto $p$; this is a well-defined map since every edge not equal to $a_i$ maps to $p$ and the edge $a_i$ stays constant throughout our map. This is moreover a continuous map since an open subset $U$ of $a_i$ has preimage of a union of open vertical slices of the right square if it does not contain $p$, and if it does, has preimage another union of open vertical slices unioned with the rest of the polygon on the left.
\end{proof}

\begin{problem}
  Consider two arcs $\alpha$ and $\beta$ embedded in $D^2 \times I$ as shown in the figure. The loop $\gamma$ is obviously nullhomotopic in $D^2 \times I$, but show that there is no nullhomotopy of $\gamma$ in the complement of $\alpha \cup \beta$.
\end{problem}
\begin{proof}
  Recall that we have the following diagram for $X$:
  \begin{center}
    \begin{tikzpicture}
      \node[anchor=north west] (1210) at (0,0) {\includegraphics{1210.pdf}};
      \draw [decorate,decoration={brace,amplitude=10pt},yshift=-2pt] (0.2,0) -- (1.8,0) node[midway,anchor=south,yshift=0.3cm] {$A_1$};
      \draw [decorate,decoration={brace,amplitude=10pt},yshift=2pt] (3.3,-2) -- (1.7,-2) node[midway,anchor=north,yshift=-0.3cm] {$A_2$};
    \end{tikzpicture}
  \end{center}
  where $A_1,A_2$ describe the range of the two open sets we use for our decomposition for van Kampen's theorem. Omitting the cylinder for clarity, we get that the three sets $A_1,A_1 \cap A_2,A_2$ are represented by the ``missing'' segments as follows:
  \begin{center}
    \begin{tikzpicture}[line cap=round,line join=round,>=triangle 45,y=1cm,x=1.3cm]
      \path
        (-4,3) edge node[anchor=south] {$c_1$} (-1,3)
        (-4,0) edge node[anchor=north] {$a_1$} (-1,0)
        (-1,1) edge node[anchor=north,yshift=2pt] {$b_1$} (-2,1)
        (-1,2) edge node[anchor=south,yshift=-2pt] {$b_1^{-1}$} (-2,2);
      \draw (-2,1) .. controls(-2.5,1) and (-2.5,2) .. (-2,2);
      \path
        (0,3) edge (2,3)
        (0,0) edge (2,0)
        (2,1) edge (0,1)
        (2,2) edge (0,2);
      \draw (0,3) node[anchor=east] {$c_1$};
      \draw (0,2) node[anchor=east] {$b_1^{-1}$};
      \draw (0,1) node[anchor=east] {$b_1$};
      \draw (0,0) node[anchor=east] {$a_1$};
      \draw (2,3) node[anchor=west] {$b_2^{-1}$};
      \draw (2,2) node[anchor=west] {$c_2$};
      \draw (2,1) node[anchor=west] {$a_2$};
      \draw (2,0) node[anchor=west] {$b_2$};
      \path
        (3,3) edge node[anchor=south] {$c_2$} (6,3)
        (3,0) edge node[anchor=north] {$a_2$} (6,0)
        (3,1) edge node[anchor=north,yshift=2pt] {$b_2$} (4,1)
        (3,2) edge node[anchor=south,yshift=-2pt] {$b_2^{-1}$} (4,2);
      \draw (4,1) .. controls(4.5,1) and (4.5,2) .. (4,2);
      \draw (-2.5,-0.5) node[anchor=north] {$A_1$};
      \draw (1,-0.5) node[anchor=north] {$A_1 \cap A_2$};
      \draw (4.5,-0.5) node[anchor=north] {$A_2$};
    \end{tikzpicture}
  \end{center}
  Note that the identifications given by the line segments for $A_1 \cap A_2$ in the center follow from following the path of the loops in our original picture, and that we get inverses because any loop around, say, $b_1$ is homotopy equivalent to one around $b_1^{-1}$ looping the other way, just by moving the loop around the horseshoe-shaped segment labeled $b_1,b_1^{-1}$. Since we can deformation retract each of $A_1$, $A_1 \cap A_2$, and $A_2$ to circles going around each of the labeled segments, we see that the loops around the segments $a_i,b_i,c_i$ act as our generators for the fundamental groups $\pi_1(A_1),\pi_1(A_1 \cap A_2),\pi_1(A_2)$ (note that we omit base points since all our sets are path connected, and so the fundamental groups are not dependent on the choice of base point). Now we see that
  \begin{gather*}
    \pi_1(A_1) = \braket{a_1,b_1,c_1} = \bigast_{i=1}^3 \mathbb{Z},\\
    \pi_1(A_1 \cap A_2) = \braket{a_1,b_1,b_1^{-1},c_1} = \braket{b_2,a_2,c_2,b_2^{-1}} = \bigast_{i=1}^4 \mathbb{Z},\\
    \pi_1(A_2) = \braket{a_2,b_2,c_2} = \bigast_{i=1}^3 \mathbb{Z},
  \end{gather*}
  and so by van Kampen's theorem, we have that
  \begin{equation*}
    \pi_1(X) = \frac{\pi_1(A_1) \ast \pi_1(A_2)}{\braket{\iota_{12}(\omega)\iota_{21}(\omega)^{-1}}} = \frac{\braket{a_1,a_2,b_1,b_2,c_1,c_2}}{\braket{c_1b_2,c_2b_1,b_1a_2^{-1},a_1b_2^{-1}}} = \frac{\braket{a_1,a_2,c_1,c_2}}{\braket{a_1c_1,a_2c_2}}.
  \end{equation*}
  \par Now we consider the loop $\gamma$ in our original picture, which is in the region $A_1 \cap A_2$. Drawn looking at a vertical cross-section of our original cylinder, we see that the $\gamma$ loops around the segments of $A_1 \cap A_2$ in the following fashion:
  \begin{center}
    \begin{tikzpicture}[line cap=round,line join=round,>=triangle 45,x=1cm,y=1cm]
      \draw[decoration={markings, mark=at position 0.375 with {\arrow{<}}}, postaction={decorate}] (0,0) circle (2);
      \fill (0,1) circle (0.05);
      \fill (0,-1) circle (0.05);
      \fill (1,0) circle (0.05);
      \fill (-1,0) circle (0.05);
      \draw (-1.4,1.4) node[anchor=south east] {$\gamma$};
      \draw (0,1) node[anchor=south] {$c_1$};
      \draw (0,-1) node[anchor=north] {$a_1$};
      \draw (1,0) node[anchor=west] {$b_1^{-1}$};
      \draw (-1,0) node[anchor=east] {$b_1$};
    \end{tikzpicture}
  \end{center}
  Choosing our base point $x_0$ to be the center of this circle, we then have that $\gamma$ is homotopy equivalent to the loop $a_1b_1c_1b_1^{-1}$ in the diagram below:
  \begin{center}
    \begin{tikzpicture}[line cap=round,line join=round,>=triangle 45,x=1cm,y=1cm]
      \fill (0,1) circle (0.05);
      \fill (0,-1) circle (0.05);
      \fill (1,0) circle (0.05);
      \fill (-1,0) circle (0.05);
      \fill (0,0) circle (0.1);
      \draw (0,0) node[anchor=north,yshift=-3pt] {$x_0$};
      \draw (0,1) node[anchor=south] {$c_1$};
      \draw (0,-1) node[anchor=north] {$a_1$};
      \draw (1,0) node[anchor=west] {$b_1^{-1}$};
      \draw (-1,0) node[anchor=east] {$b_1$};
      \pgfplothandlerlineto
      \pgfplotfunction{\x}{0, 1, ..., 360}{
          \pgfpointxy     
          {2*cos(\x) * sin(2 * \x + 90)}
          {2*sin(\x) * sin(2 * \x + 90)}
      }
      \pgfusepath{stroke}
      \path (0.05,2) edge[->] (0.1,2);
      \path (-0.05,-2) edge[->] (-0.1,-2);
      \path (-2,0.05) edge[->] (-2,0.1);
      \path (2,-0.05) edge[->] (2,-0.1);
    \end{tikzpicture}
  \end{center}
  Since $a_1b_1c_1b_1^{-1} = a_1a_2c_1a_2^{-1} \ne 1$ in $\pi_1(X)$, $\gamma$ is not represented by $1 \in \pi_1(X)$, and so is not nulhomotopic in $X$.
\end{proof}

\setcounter{subsubsection}{13}
\begin{problem}
  Consider the quotient space of a cube $I^3$ obtained by identifying each square face with the opposite face via the right-handed screw motion consisting of a translation by one unit in the direction perpendicular to the face combined with a one-quarter twist of the face about its center point. Show that this quotient space $X$ is a cell complex with two $0$-cells, four $1$-cells, three $2$-cells, and one $3$-cell. Using this structure, show that $\pi_1(X)$ is the quaternion group $\{\pm1,\pm i,\pm j,\pm k\}$, of order eight.
\end{problem}
\begin{proof}
  We first draw the $2$-cell with the given identifications:
  \begin{center}
    \begin{tikzpicture}[line cap=round,line join=round,>=stealth,x=2cm,y=2cm]
      %back square
      \path
        (0,0) edge[-<-=0.52] (0,2)
        (0,2) edge[->-=0.52] node[anchor=south] {$a$} (2,2)
        (2,2) edge[-<-=0.52] node[anchor=west] {$b$} (2,0)
        (2,0) edge[->-=0.52] node[anchor=north] {$c$} (0,0)
        ;
      %front square
      \path
        (-.75,1.25) edge[-<-=0.52,commutative diagrams/crossing over] node[anchor=south] {$b$} (1.25,1.25)
        (1.25,1.25) edge[->-=0.52,commutative diagrams/crossing over] node[anchor=west] {$c$} (1.25,-0.75)
        (1.11,1.25) edge (1.25,1.25)
        (-.75,-.75) edge[->-=0.52] node[anchor=east] {$a$} (-.75,1.25)
        (1.25,-0.75) edge[-<-=0.52] node[anchor=north] {$d$} (-.75,-.75)
        ;       
      %connecting segments
      \path
        (0,0) edge[-<-=0.52] node[anchor=north west,yshift=4pt] {$b$} (-.75,-.75)
        (0,2) edge[->-=0.52] node[anchor=south east] {$c$} (-.75,1.25)
        (2,2) edge[-<-=0.52] node[anchor=south east] {$d$} (1.25,1.25)
        (2,0) edge[->-=0.52] node[anchor=north west] {$a$} (1.25,-0.75)
        ;
      \draw (0,1) node[anchor=east] {$d$};
      \fill (-.75,-.75) circle (0.05);
      \fill (1.25,1.25) circle (0.05);
      \fill (2,0) circle (0.05);
      \fill (0,2) circle (0.05);
      \draw (-.75,1.25) node {$\bigstar$};
      \draw (1.25,-.75) node {$\bigstar$};
      \draw (0,0) node {$\bigstar$};
      \draw (2,2) node {$\bigstar$};
    \end{tikzpicture}
  \end{center}
  where edges with the same label are identified, the vertices $\bullet$ are identified, as are the vertices $\bigstar$, and opposite faces are identified. These edge identifications follow just by considering opposite faces and using the one-quarter twist identification, and by matching heads/tails of edges to the vertices. Since this cube is then filled with a 3-cell, we see that $X$ is a cell complex with $0$-cells, four $1$-cells, three $2$-cells, and one $3$-cell.
  \par Now we calculate the fundamental group. Recall that attaching $n$-cells for $n>2$ in a CW decomposition has no effect on the fundamental group, so we only have to consider the fundamental group for the $2$-skeleton drawn above.
  \par To calculate this fundamental group, we first deformation retract $d$ to a point. Thus, our 2-cells attach according to the schema $ac^{-1}b^{-1}$, $c^{-1}ba^{-1}$, and $bc^{-1}a$, and we have the presentation
  \begin{equation*}
    \pi_1(X) = \frac{\braket{a,b,c}}{\braket{ac^{-1}b^{-1}, c^{-1}ba^{-1}, bc^{-1}a}}.
  \end{equation*}
  Letting $a = i$, $b = j$, $c = k$ this is equivalent to the presentation
  \begin{equation*}
    \pi_1(X) = \braket{i,j,k | i = jk,~ j = ki,~ ij = k}.
  \end{equation*}
  We can multiply $i = jk$ by $i$ and $ij = k$ by $k$. Substituting $i = jk$ into $j = ki$ to get $j = kjk$, multiplying by $j$ to get $j^2 = jkjk$, and then substituting $i = jk$ again to get $j^2 = ijk$, we finally have the presentation
  \begin{equation*}
    \pi_1(X) = \braket{i,j,k | i^2 = j^2 = k^2 = ijk},
  \end{equation*}
  which is the usual presentation for the quaternion group.
\end{proof}

\printbibliography
\cleardoublepage
\pdfbookmark[1]{List of Solved Exercises}{det}
\renewcommand*\contentsname{List of Solved Exercises}
{\footnotesize\tableofcontents}
\end{document}
